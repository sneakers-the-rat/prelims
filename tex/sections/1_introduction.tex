%!TEX root = ../prelims_main.tex
% \documentclass[../prelims_main.tex]{subfiles}

% \begin{document}

\section{Introduction}

\subsection{Processing Ill-defined Phonetic Categories}

\begin{itemize}
	\item auditory processing as domain-general and domain-specific across multiple timescales \cite{norman-haignereHierarchicalIntegrationMultiple2020}
\end{itemize}

\subsection{paradoxes}

levels of analysis:

phonetic perception has paradoxes at several levels of analysis that are not mutually discrete.

\textbf{ontic/algorithmic}: what \textit{are} phonemes? are they positive descriptions of combinations of features, or negative descriptions of forbidden spectrotemporal state transitions?

\textbf{implementation}: to some degree the methodological and theoretical disagreements between the feature-detection and population-computation models of phonetic perception mirror the single-cell/multicellular computation dichotomy described in the introduction of \cite{dubreuilComplementaryRolesDimensionality2020}. 

\begin{itemize}
	\item speed of processing vs. variability within category
	\item neurons that process auditory information at phonetic timescales are relatively insensitive to spectral quality \cite{norman-haignereHierarchicalIntegrationMultiple2020}
\end{itemize}

\subsection{<some of that neural theories of phonetic processing}

\begin{itemize}
\item why are auditory neurons potentialyl sensitive to multiple stimulus features/how does that contribute to generalizable ill-defined catgories? \cite{macellaioWhySensoryNeurons2020}
\item abrupt transitions, at least in neural data \cite{durstewitzAbruptTransitionsPrefrontal2010}
\item other reward-learning regions like RSC \cite{millerRetrosplenialCorticalRepresentations2019}
\item multimodal representations and preserved neural manifold dynamics across inference tasks in M1 \cite{gallegoCorticalPopulationActivity2018}
\item timescales of processing expand across auditory hierarchy (and more generally have different timescales of integration and lags) \cite{norman-haignereHierarchicalIntegrationMultiple2020} and are lateralized \cite{levyCircuitAsymmetriesUnderlie2019a}
\end{itemize}

\subsection{scraps}

\begin{itemize}
\item theoretical problems with simplified stimuli - low-dimensional and linearly-separable stimulus spaces are fundamentally different than the high complexity of naturalistic stimuli... for all we know the computations are just straight up not comparable! \cite{schuesslerInterplayRandomnessStructure2020}

\end{itemize}

% \end{document}