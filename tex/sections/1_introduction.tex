%!TEX root = ../prelims_main.tex
% \documentclass[../prelims_main.tex]{subfiles}

% \begin{document}

\section{Introduction}
\begin{multicols}{2}
\subsection{Phonemes are Language Games}

\begin{leftbar}

"Consider for example the proceedings that we call "games". [...] For if you look at them you will not see something that is common to all, but similarities, relationships, and a whole series of them at that. [...] Are they all 'amusing'? Compare chess with noughts and crosses. Or is there always winning and losing, or competition between players? Think of patience. [...] Look at the parts played by skill and luck; and at the difference between skill in chess and skill in tennis. 

And the result of this examination is: we see a complicated network
of similarities overlapping and criss-crossing: sometimes overall similarities, sometimes similarities of detail. [...] And we extend our concept as in spinning a thread we twist fibre on fibre. And the strength of the thread does not reside in the fact that some one fibre runs through its whole length, but in the overlapping of many fibres."

\textit{-Wittgenstein, Philosophical Investigations: 66-67\cite{wittgensteinPhilosophicalInvestigations1968}}

\end{leftbar}

Cognitive reality is characterized by its discreteness: rather than a continuous undifferentiated gradient wash of sensation and cognition, we experience objects, concepts, and thoughts. Speech is a continuous, high-dimensional, high-variability acoustic signal, yet it is perceived as a small number of discrete phonemes. The acoustic structure of phonemes is a sort of "Family Resemblance"\cite{wittgensteinPhilosophicalInvestigations1968} --- the truly extravagant variability of speech has thus far defied any simple, definite acoustic parameterization of its phonemes. 

Category representation theories are intimately related (and occasionally literally isometric to \cite{Edelman1998}) to theories of the measurement of similarity, which is dominated by geometric models\cite{Tversky1977}. nearly universally presuppose that categories exist in a feature space such that there exist some number of features that describe each instance of an object to be categorized.

So the determination of *what* the features are is of utmost importance. Traditional phonetics experiments attempt to parameterize a phoneme by producing synthetic sounds that vary parametrically across some feature. They then present these stimuli to people and 

The history of this question includes Shepard and Tversky's multidimensional scaling and its criticisms, and also extends through Shepherds' "second-order isomorphisms" (cite representation is representation of similarity)

\begin{itemize}
	\item Short description of phonetic acoustics, why they're games
	\item General statement on importance of understanding neural implementation of a game-recognition system
	\item The natural analog of the philosophical problem of universals in the conditioning paradigm is stimulus generalization \cite{roschWittgensteinCategorizationResearch1987}
	\item parameterized vs natural speech is actually reflective of a much larger positivist/naturalist philosophical divide -- they presuppose by testing a parameter of category membership, but postiive evidence is not evidence that parameter is actually constitutive of the category itself -- for example if you had two categories "games" and "cars," "weight" might be a reasonably good way to assign category membership, but it is not at all the only, or even the most salient difference between those categories. 
	\item indeed feature hierarchies from phonetics belie the utility of parameterized stimuli
	\item and categories are necessarily only defined in reference to one another, they participate in a `category space' -- so if you construct a category space with only one sensible axis between them you truly are not modeling the problem.
	\item animals use family resemblance of multiple features even when there is a single dimension that is perfectly informative of category membership \cite{leaUseMultipleDimensions2008, couchmanRulesResemblanceTheir2010}
	\item assuming feature dimensions is always a bad assumption -- eg what features have the metric structure that measure similarity/dissimilarity of rectangles? \cite{krantzSimilarityRectanglesAnalysis1975a}
	\item end with three uh impacts: 1) observe how auditory system learns complex categories, 2) resolve questions in phonetics re: what phonemes "are," and 3) contribute to fundamental questions faced by all neural systems: how are the building blocks of sensory discretization that defines all our perceptual and cognitive systems learned and used? -- the gap of *implementation* is actually critical, seeing how a neural system learns attributes and categories could resolve 

\end{itemize}

\subsection{Learning to play a Language Game}

\begin{itemize}
	\item Cognitive categorization learning mostly operates as family resemblances that have incomplete/nonplatonic feature sets that unite them (\cite{roschFamilyResemblancesStudies1975}\cite{roschWittgensteinCategorizationResearch1987} \cite{couchmanRulesResemblanceTheir2010})
	\item tversky talked about this in terms of set theory and resemblance \cite{tverskyStudiesSimilarity1978} \cite{Tversky1970}
	\item The notion of what even constitutes a "features" is ill-defined (history of phonetics stuff about how we've just been basically trying to reverse-engineer these)
	\item The learning problem is one where the listener needs to learn *both* what constitutes a category *and* the features that are useful for determining category. (review some of caitlin's infant speech learning stuff)
\end{itemize}

\subsection{paradoxes}

levels of analysis:

phonetic perception has paradoxes at several levels of analysis that are not mutually discrete.

\textbf{ontic/algorithmic}: what \textit{are} phonemes? are they positive descriptions of combinations of features, or negative descriptions of forbidden spectrotemporal state transitions?

\textbf{implementation}: to some degree the methodological and theoretical disagreements between the feature-detection and population-computation models of phonetic perception mirror the single-cell/multicellular computation dichotomy described in the introduction of \cite{dubreuilComplementaryRolesDimensionality2020}. 

\begin{itemize}
	\item speed of processing vs. variability within category
	\item neurons that process auditory information at phonetic timescales are relatively insensitive to spectral quality \cite{norman-haignereHierarchicalIntegrationMultiple2020}
\end{itemize}

\subsection{<some of that neural theories of phonetic processing}

\begin{itemize}
\item auditory processing as domain-general and domain-specific across multiple timescales \cite{norman-haignereHierarchicalIntegrationMultiple2020}
\item why are auditory neurons potentialyl sensitive to multiple stimulus features/how does that contribute to generalizable ill-defined catgories? \cite{macellaioWhySensoryNeurons2020}
\item abrupt transitions, at least in neural data \cite{durstewitzAbruptTransitionsPrefrontal2010}
\item other reward-learning regions like RSC \cite{millerRetrosplenialCorticalRepresentations2019}
\item multimodal representations and preserved neural manifold dynamics across inference tasks in M1 \cite{gallegoCorticalPopulationActivity2018}
\item timescales of processing expand across auditory hierarchy (and more generally have different timescales of integration and lags) \cite{norman-haignereHierarchicalIntegrationMultiple2020} and are lateralized \cite{levyCircuitAsymmetriesUnderlie2019a}
\end{itemize}

probs w/ discriminatory models: how is the comparison done? eg. you could start learning features by just comparing every x thing with y thing, but then you would have to hold some representation of each in order to compare. 

\subsection{scraps}

\begin{itemize}
\item theoretical problems with simplified stimuli - low-dimensional and linearly-separable stimulus spaces are fundamentally different than the high complexity of naturalistic stimuli... for all we know the computations are just straight up not comparable! \cite{schuesslerInterplayRandomnessStructure2020}

\end{itemize}

\end{multicols}

% \end{document}