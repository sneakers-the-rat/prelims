%!TEX root = ../prelims_main.tex
% \documentclass[../prelims_main.tex]{subfiles}

% \begin{document}

\section{Introduction}
\begin{multicols}{2}
\subsection{Phonemes are Language Games}

\begin{leftbar}

"Consider for example the proceedings that we call "games". [...] For if you look at them you will not see something that is common to all, but similarities, relationships, and a whole series of them at that. [...] Are they all 'amusing'? Compare chess with noughts and crosses. Or is there always winning and losing, or competition between players? Think of patience. [...] Look at the parts played by skill and luck; and at the difference between skill in chess and skill in tennis. 

And the result of this examination is: we see a complicated network
of similarities overlapping and criss-crossing: sometimes overall similarities, sometimes similarities of detail. [...] And we extend our concept as in spinning a thread we twist fibre on fibre. And the strength of the thread does not reside in the fact that some one fibre runs through its whole length, but in the overlapping of many fibres."

\textit{-Wittgenstein, Philosophical Investigations: 66-67\cite{wittgensteinPhilosophicalInvestigations1968}}

\end{leftbar}

\begin{itemize}
	\item Short description of phonetic acoustics, why they're games
	\item General statement on importance of understanding neural implementation of a game-recognition system
\end{itemize}

\subsection{Learning to play a Language Game}

\begin{itemize}
	\item Cognitive categorization learning mostly operates as family resemblances that have incomplete/nonplatonic feature sets that unite them (\cite{roschFamilyResemblancesStudies1975}\cite{roschWittgensteinCategorizationResearch1987} \cite{couchmanRulesResemblanceTheir2010})
	\item tversky talked about this in terms of set theory and resemblance \cite{tverskyStudiesSimilarity1978} \cite{Tversky1970}
	\item The notion of what even constitutes a "features" is ill-defined (history of phonetics stuff about how we've just been basically trying to reverse-engineer these)
	\item The learning problem is one where the listener needs to learn *both* what constitutes a category *and* the features that are useful for determining category. (review some of caitlin's infant speech learning stuff)
\end{itemize}

\subsection{paradoxes}

levels of analysis:

phonetic perception has paradoxes at several levels of analysis that are not mutually discrete.

\textbf{ontic/algorithmic}: what \textit{are} phonemes? are they positive descriptions of combinations of features, or negative descriptions of forbidden spectrotemporal state transitions?

\textbf{implementation}: to some degree the methodological and theoretical disagreements between the feature-detection and population-computation models of phonetic perception mirror the single-cell/multicellular computation dichotomy described in the introduction of \cite{dubreuilComplementaryRolesDimensionality2020}. 

\begin{itemize}
	\item speed of processing vs. variability within category
	\item neurons that process auditory information at phonetic timescales are relatively insensitive to spectral quality \cite{norman-haignereHierarchicalIntegrationMultiple2020}
\end{itemize}

\subsection{<some of that neural theories of phonetic processing}

\begin{itemize}
\item auditory processing as domain-general and domain-specific across multiple timescales \cite{norman-haignereHierarchicalIntegrationMultiple2020}
\item why are auditory neurons potentialyl sensitive to multiple stimulus features/how does that contribute to generalizable ill-defined catgories? \cite{macellaioWhySensoryNeurons2020}
\item abrupt transitions, at least in neural data \cite{durstewitzAbruptTransitionsPrefrontal2010}
\item other reward-learning regions like RSC \cite{millerRetrosplenialCorticalRepresentations2019}
\item multimodal representations and preserved neural manifold dynamics across inference tasks in M1 \cite{gallegoCorticalPopulationActivity2018}
\item timescales of processing expand across auditory hierarchy (and more generally have different timescales of integration and lags) \cite{norman-haignereHierarchicalIntegrationMultiple2020} and are lateralized \cite{levyCircuitAsymmetriesUnderlie2019a}
\end{itemize}

\subsection{scraps}

\begin{itemize}
\item theoretical problems with simplified stimuli - low-dimensional and linearly-separable stimulus spaces are fundamentally different than the high complexity of naturalistic stimuli... for all we know the computations are just straight up not comparable! \cite{schuesslerInterplayRandomnessStructure2020}

\end{itemize}

\end{multicols}

% \end{document}