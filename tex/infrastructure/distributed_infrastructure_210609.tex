\documentclass{article}

\usepackage{jls_base}
\title{Distributed Infrastructure For Systems Neuroscience Would Revolutionize The Discipline}
\author{Jonny L. Saunders}
\date{\today}
\addbibresource{/Users/jonny/git/sneakers-the-rat.github.io/_preblog/assets/blog.bib}
\begin{document}
\maketitle
\tableofcontents
\clearpage
\begin{leftbar}
If we can make something decentralised, out of control, and of great
simplicity, we must be prepared to be astonished at whatever might grow
out of that new medium.

\href{https://www.w3.org/1998/02/Potential.html}{Tim Berners-Lee (1998):
Realising the Full Potential of the Web}
\end{leftbar}

\begin{leftbar}
A good analogy for the development of the Internet is that of constantly
renewing the individual streets and buildings of a city, rather than
razing the city and rebuilding it. The architectural principles
therefore aim to provide a framework for creating cooperation and
standards, as a small ``spanning set'' of rules that generates a large,
varied and evolving space of technology.

\href{https://datatracker.ietf.org/doc/html/rfc1958}{RFC 1958:
Architectural Principles of the Internet}
\end{leftbar}

\begin{leftbar}
In building cyberinfrastructure, the key question is not whether a
problem is a ``social'' problem or a ``technical'' one. That is putting
it the wrong way around. The question is whether we choose, for any
given problem, a primarily social or a technical solution

\href{https://doi.org/10.1007/978-1-4020-9789-8_5}{Bowker, Baker,
Millerand, and Ribes (2010): Toward Information Infrastructure Studies}
\cite{bowkerInformationInfrastructureStudies2010},\end{leftbar}
\begin{leftbar}
The critical issue is, how do actors establish generative platforms by
instituting a set of control points acceptable to others in a nascent
ecosystem? \cite{tilsonDigitalInfrastructuresMissing2010},\end{leftbar}
Acknowledgements (make sure to double check spelling!!!): * Lauren E.
Wool * Gaby Hayden * Eartha Mae * jakob voigts for participating in the
glue wiki * nwb \& dandi team for dealing w/ my inane rambling * open
behavior team * metascience class for some of these ideas \textless3 *
mike for letting me always go rogue * os \& avery for STS recs * add
rest from presentation

\hypertarget{introduction}{%
\section{Introduction}\label{introduction}}

(initial sentence re: we use computers all day and it's really hard and
takes up all of our time. make sure to specify for systems neuro.)

We work in technical islands that range from individual researchers, to
labs, consortia, and at their largest a few well-funded organizations.
Our knowledge dissemination systems are as nimble as the static pdfs and
ephemeral conference talks that they have been for decades (save for the
godforsaken Science Twitter that we all correctly love to hate).
Experimental instrumentation except for that at the polar extremes of
technological complexity or simplicity is designed and built custom,
locally, and on-demand. Software for performing experiments is a
patchwork of libraries that satisfy some of the requirements of the
experiment, sewn together by some uncommented script written years ago
by a grad student who left the lab long-since. The technical knowledge
to build both instrumentation and software is fragmented and unavailable
as it sifts through the funnels of word-limited methods sections and
never-finished documentation. And O Lord Let Us Pray For The Data, born
into this world without coherent form to speak of, indexable only by
passively-encrypted notes in a paper lab notebook, dressed up for the
analytical ball once before being mothballed in ignominy on some
unlabeled external drive.

In sum, all the ways our use and relations with computers are
idiosyncratic and improvised are not isolated, but a symptom of a
broader deficit in \textbf{digital infrastructure} for science. The
yawning mismatch between our ambitions of what digital technology
\emph{should} allow us to do and the state of digital infrastructure
hints at the magnitude of the problem: the degree to which the symptoms
of digital deinfrastructuring define the daily reality of science is
left as an exercise to the reader.

If the term infrastructure conjures images of highways and plumbing,
then surely digital infrastructure would be flattered at the
association. By analogy they illustrate many of its promises and
challenges: when designed to, it can make practically impossible things
trivial, allowing the development of cities by catching water where it
lives and snaking it through tubes and tunnels sometimes directly into
your kitchen. Its absence or failure is visible and impactful, as in the
case of power outages. There is no guarantee that it ``optimally''
satisfies some set of needs for the benefit of the greatest number of
people, as in the case of the commercial broadband duopolies. It exists
not only as its technical reality, but also as an embodied and shared
set of social practices, and so even when it does exist its final form
is not inevitable or final; as in the case of bottled water producers
competing with municipal tap water on a behavioral basis despite being
dramatically less efficient and more costly. Finally it is not socially
or ethically neutral, where the impact of failure to build or maintain
it is not equally shared, as in the expression of institutional racism
that was the Flint, Michigan water crisis \cite{michicancivilrightscommissionFlintWaterCrisis2017},.
Being digitally deinfrastructured is not our inevitable and eternal
fate, but the course of infrastructuring is far from certain. It is not
the case that ``scientific digital infrastructure'' will rise from the
sea monolithically as a natural result of more development time and
funding, but instead has many possible futures\cite{mirowskiFutureOpenScience2018},,each with their own advocates and
beneficiaries. Without concerted and strategic counterdevelopment based
on a shared and liberatory ethical framework, science is poised to
follow other domains of digital technology down the dark road of
platform capitalism. The prize of owning the infrastructure that the
practice of science is built on is too great, and it is not hard to
imagine tech behemoths buying out the emerging landscape of small
scientific-software-as-a-service startups and selling subscriptions to
Science Prime.

This paper is an argument that \textbf{decentralized} digital
infrastructure is the best means of realizing the promise of digital
technology for science. I will draw from several disciplines and
knowledge communities like Science and Technology Studies (STS), Library
and Information Science, open source software developers, and internet
pirates, among others to articulate a vision of an infrastructure in
three parts: \textbf{shared data, shared tools, and shared knowledge.} I
will start with a brief description of what I understand to be the state
of our digital infrastructure and the structural barriers and incentives
that constrain its development. I will then propose a set of design
principles for decentralized infrastructure and possible means of
implementing it. I will close with contrasting visions of what science
could be like depending on the course of our infrastructuring, and my
thoughts on how different actors in the scientific system can contribute
to and benefit from decentralization.

I insist that what I will describe is \emph{not utopian} but is
eminently practical --- with a bit of development to integrate and
improve them, \textbf{everything I propose here already exists and is
widely used.} A central principle of decentralized systems is embracing
heterogenaeity: harnessing the power of the diverse ways we do science
instead of constraining them. Rather than a patronizing argument that
everyone needs to fundamentally alter the way they do science, the
systems that I describe are specifically designed to be easily
incorporated into existing practices and adapted to variable needs. In
this way decentralized systems are arguably \emph{more practical} than
the dream that one system will be capable of expanding to the scale of
all science --- and as will hopefully become clear, inarguably
\emph{more powerful} than a disconnected sea of platforms and services.

An easy and common misstep is to categorize this as solely a
\emph{technical} challenge. Instead the challenge of infrastructure is
also \emph{social} and \emph{cultural} --- it involves embedding any
technology in a set of social practices, a shared belief that such
technology exists and that its form it not neutral, and a sense of
communal valuation and purpose.

The social and technical perspectives are both essential, but make some
conflicting demands on the construction of the piece: Infrastructuring
requires considering the interrelatedness and mutual reinforcement of
the problems to be addressed, rather than treating them as isolated
problems that can be addressed piecemeal with a new package. Such a
broad scope trades off with a detailed description of the relevant
technology and systems, but a myopic techno-zealotry that does not
examine the social and ethical nature of scientific practice risks
reproducing or creating new sources of harm. As a balance I will not be
proposing a complete technical specification or protocol, but describing
the general form of the tools and some existing examples that satisfy
them; I will not attempt a full history or treatment of the problem of
infrastructuring, but provide enough to motivate the form of the
proposed implementations.

My understanding of this problem is, of course, uncorrectably structured
by the horizon of disciplines around systems neuroscience that has
preoccupied my training. While the core of my argument is intended to be
a sketch compatible with sciences and knowledge systems generally, my
examples will sample from and my focus will skew to my experience. In
many cases, my use of ``science'' or ``scientist'' could be
``neuroscience'' or ``neuroscientist,'' but I will mostly use the former
to avoid the constant context switches. I ask the reader for a measure
of patience for the many ways this argument requires elaboration and
modification for distant fields.

\hypertarget{the-state-of-things}{%
\section{The State of Things}\label{the-state-of-things}}

\hypertarget{the-costs-of-being-deinfrastructured}{%
\subsection{The Costs of being
Deinfrastructured}\label{the-costs-of-being-deinfrastructured}}

Framing the many challenges of scientific digital technology development
as reflective of a general digital infrastructure deficit gives a shared
etiology to the technical and social harms that are typically treated
separately. It also allows us to problematize other symptoms that are
embedded in the normal practice of contemporary science.

To give a sense of the scale of need for digital scientific
infrastructure, as well as a general scope for the problems the proposed
system is intended to address, I will list some of the present costs.
These lists are grouped into rough and overlapping categories, but make
no pretense at completeness and have no particular order.

Impacts on the \textbf{daily experience} of researchers include:

\begin{itemize}

\item
  A prodigious duplication and dead-weight loss of labor as each lab,
  and sometimes each person within each lab, will reinvent basic code,
  tools, and practices from scratch. Literally it is the inefficiency of
  the
  \href{https://en.wikipedia.org/wiki/Deadweight_loss\#Harberger's_triangle}{Harberger's
  triangle} in the supply and demand system for scientific
  infrastructure caused by inadequate supply. Labs with enough resources
  are forced to pay from other parts of their grants to hire
  professional programmers and engineers to build the infrastructure for
  their lab (and usually their lab or institute only), but most just
  operate on a purely amateur basis. Many PhD students will spend the
  first several years of their degree re-solving already-solved
  problems, chasing the tails of the wrong half-readable engineering
  whitepapers, in their 6th year finally discovering the technique that
  they actually needed all along. That's not an educational or training
  model, it's the effect of displacing the undone labor of unbuilt
  infrastructure on vulnerable graduate workers almost always paid
  poverty wages.
\item
  At least the partial cause of the phenomenon where ``every scientist
  needs to be a programmer now'' as people who aren't particularly
  interested in being programmers --- which is \emph{fine} and
  \emph{normal} --- need to either suffer through code written by some
  other unlucky amateur or learn an entire additional discipline in
  order to do the work of the one they chose. Because there isn't more
  basic scientific programming infrastructure, everyone needs to be a
  programmer.
\item
  A great deal of pain and alienation for early-career researchers
  (ECRs) not previously trained in programming before being thrown in
  the deep end. Learning data hygeine practices like backup, annotation,
  etc. ``the hard way'' through some catastrophic loss is accepted myth
  in much of science. At some scale all the very real and widespread
  pain, and guilt, and shame felt by people who had little choice but to
  reinvent their own data management system must be recognized as an
  infrastructural, rather than a personal problem.
\item
  The high cost of ``openness'' and the dearth of data transparency. It
  is still rare for systems neuroscience papers to publish the full, raw
  data along with all the analysis code, often because (in addition to
  the extraordinarily meagre incentives to do so) the data \emph{and}
  analysis code are both completely homebrew and often omitted just due
  to the labor of cleaning it or the embarassment of sharing
  it\footnote{which, to be clear, is a valid feeling and is reflective
    of a failure of infrastructure, not a personal failure.}. The ``Open
  science'' movement, roughly construed, has made a holy mess of the
  social climate around openness, publicly shaming ``closed scientists''
  on leaderboards and only occasionally recalling the relative luxury of
  labor or expertise to become ``open.'' ``Openness'' is not a uniform
  or universal goal for all science, but for those for whom it makes
  sense, we need to provide the appropriate tooling before insisting on
  a change in scientific norms. We can't expect data transparency from
  researchers while it is still so \emph{hard.}
\end{itemize}

Impacts on the \textbf{system of scientific inquiry} include:

\begin{itemize}

\item
  A profoundly leaky knowledge acquisition system where entire PhDs
  worth of data can be lost and rendered useless when a student leaves a
  lab and no one remembers how to access the data or how it's formatted.
\item
  The inevitability of continual replication crises because it is often
  literally impossible to replicate an experiment that is done on a rig
  that was built one time, used entirely in-lab code, and was never
  documented
\item
  A perhaps doomed intellectual endeavor as we attempt to understand the staggering
  complexity of the brain by peering at the brain through the
  pinprickiest peephole of just the most recent data you or your lab
  have collected rather than being able to index across all relevant
  data from not only your lab, but all other labs that have measured the
  same phenomena. The unnecessary reduplication of experiments becomes
  not just a methodological limitation, but an ethical catastrophe as
  researchers have little choice but to abandon the elemental principle
  of sacrificing as few animals as possible to understand a phenomenon.
\item
  A hierarchy of prestige that devalues the labor of multiple groups of
  technicians, animal care workers, and so on. Authorship is the coin of
  the realm, but many researchers that do work fundamental to the
  operation of science only receive the credit of an acknowledgement. We
  need a system to value and assign credit for the immense amount of
  technical and practical knowledge and labor they produce.
\end{itemize}

Impacts on the relationship between \textbf{science and society}:

\begin{itemize}

\item
  An insular system where the inaccessibility of all the ``contextual''
  knowledge \cite{woolKnowledgeNetworksHow2020,
  barleyBackroomsScienceWork1994} that is beneath the level of
  publication but necessary to perform experiments, like ``how to build
  this apparatus,'' ``what kind of motor would work here,'' etc. is a
  force that favors established and well-funded labs who can rely on
  local knowledege and hiring engineers/etc. and excludes new,
  lesser-funded labs at non-ivy institutions. The concentration of
  technical knowledge magnifies the inequity of strongly skewed funding
  distributions such that the most well-funded labs can do a completely
  different kind of science than the rest of us, turning the
  positive-feedback loop of funding begetting funding ever faster.
\item
  An absconscion with the public resources we are privileged enough to
  receive, where rather than returning the fruits of the many technical
  challenges we are tasked with solving to the public in the form of
  data, tools, collected practical knowledge, etc. we largely return
  papers, multiplying the above impacts of labor duplication and
  knowledge inaccessibility by the scale of society.
\item
  \textbf{{[}expand here on the platform capitalism argument more fully
  after hinting at it before{]}}
\end{itemize}

and so on.

Considered separately, these are problems, but together they are a
damning indictment of our role as stewards of our corner of the human
knowledge project.

We arrive at this situation not because scientists are lazy and
incompetent, but because the appropriate tools that fit the requirements
of their discipline don't exist, and traditional patterns of centralized
organization can't scale to encompass their diverse needs. There is an
enormous amount of work being done by researchers and engineers on all
of these problems, and a huge amount of progress has been made on them.
My intention is not to shame or devalue anyone's work, but to try and
describe a path towards making it mutually reinforcing.

Before proposing a potential solution to some of the above problems, it
is important to motivate why they haven't already been solved, or why
their solution is not necessarily imminent. To do that, we need a sense
of the conditions in which it is developed.

\hypertarget{systems-neuroscience-specifically}{%
\subsection{Systems Neuroscience
Specifically\ldots{}}\label{systems-neuroscience-specifically}}

Every discipline has its own particular technical needs, and is subject
to its own peculiar history and culture. Though the type of
comprehensive distributed infrastructure I will describe later is a
domain-general project, systems neuroscience specifically lacks some
features of it that are present in immediately neighboring disciplines
like genetics and cognitive psychology. I won't attempt a complete
explanation, but instead will offer a few patterns I have noticed in my
own limited exposure to the field that might serve as the beginnings of
one. I want to be very clear throughout that I am never intending to
cast shade on the work of anyone who has or does build and maintain the
scientific infrastructure that exists --- in fact the opposite, that
y'all deserve more resources.

\hypertarget{diversity-of-measurements}{%
\subsubsection{Diversity of
Measurements}\label{diversity-of-measurements}}

Molecular biology and genetics are perhaps the neighboring disciplines
with the best data sharing and analytical structure, spawning and
occupying the near totality of a new subdiscipline of Bioinformatics
(for an absolutely fascinating ethnography, see \cite{bietzCollaborationMetagenomicsSequence2009},).Though the experiments
are of course just as complex as those in systems neuroscience, most
rely on a small number of stereotyped sequencing (meta?)methods that
result in the same one-dimensional, four character sequence data
structure of base pairs. Systems neuroscience experiments increasingly
incorporate dozens of measurements, electrophysiology, calcium imaging,
multiple video streams, motion, infrared, and other sensors, and so on.
This is increasingly true as neuroscientists are attempting ever more
complex and naturalistic neuroethological experiments. Even the
seemingly ``common'' electrophysiological or multiphoton imaging data
can have multiple forms --- raw voltage traces? spike times? spike
templates and times? single or multiunit? And these forms go through
multiple intermediate stages of processing --- binning, filtering,
aggregating, etc. --- each of which could be independently valuable and
thus represented alongside their provenance in a theoretical data
schema. Mainen and colleagues note this problem as well:

\begin{leftbar}
The data sets generated by a functional neuroscience experiment are
large. They can also be complex and multimodal in ways that, say,
genomic data might not be, embracing recordings of activity, behavioural
patterns, responses to perturbations, and subsequent anatomical
analysis. Researchers have no agreed formats for integrating different
types of information. Nor are there standard systems for curating,
uploading and hosting highly multimodal data. \cite{mainenBetterWayCrack2016},\end{leftbar}
The \href{https://www.nwb.org/}{Neurodata Without Borders} project has
made a valiant effort to unify these multiple formats, but has for
reasons that I won't lay claim to knowing has yet to see widespread
adoption. Contrast this with the
\href{https://bids.neuroimaging.io/}{BIDS} data structure for fMRI data,
where by converting your data to the structure you unlock a huge library
of analysis pipelines for free. The beginnings of generalized platforms
for neuroscientific data built on top of NWB are starting to happen in
trickles and droplets, but they are still very much the exception rather
than the rule.

We should not be so proud as to believe that our data is somehow
uniquely complex. Theorizing about and reconciling the mass and
heterogeneity of data in the universe is the subject of
\href{https://en.wikipedia.org/wiki/Information_science}{multiple}
full-fledged
\href{https://en.wikipedia.org/wiki/Library_science}{disciplines}, and
the conflict between simplified and centralized \cite{bakerMaintainingDublinCore2005},andsprawling and distributed \cite{berners-leeSEMANTICWEB2001},systemsis well-trodden --- and we
should learn from it! We could instead think of the complexity of our
data and the tools we develop to address it as what we have to offer the
broader human mission towards a unified system of knowledge.

\hypertarget{diversity-of-preps}{%
\subsubsection{Diversity of Preps}\label{diversity-of-preps}}

Though there are certain well-limbered experimental backbones like the
two-alternative forced choice task, even within them there seems to be a
comparatively broad diversity of experimental preparations in systems
neuro relative to adjacent fields. Even a visual two-alternative forced
choice task is substantially different than an auditory one, but there
is almost nothing shared between those and, for example,
\href{https://doi.org/10.7554/eLife.29053}{measuring the representation
of 3d space in a free-flying echolocating bat}. So unlike cognitive
neuroscience and psychophysics that has tools like
\href{https://pavlovia.org/}{pavlovia} where the basic requirements and
structure of experiments are more standardized, BioRXiv is replete with
technical papers documenting ``high throughput systems for this one very
specific experiment'' and there
\href{https://docs.auto-pi-lot.com}{isn't} a true experimental framework
that satisfies the need for flexibility.

Mainen and colleagues note that this causes another problem distinct
from variable outcome data, the even more variable and largely
unreported metadata that parameterizes the minute details of
experimental preps:

\begin{leftbar}
Worse, neuroscientists lack standardized vocabularies for describing the
experimental conditions that affect brain and behavioural functions.
Such a vocabulary is needed to properly annotate functional neural data.
For instance, even small differences in when a water drop is released
can affect how a mouse's brain processes this event, but there is no
standard way to specify such aspects of an experiment. \cite{mainenBetterWayCrack2016},\end{leftbar}
The problem of universal annotation and metadata reporting can be
reframed, not as a \emph{barrier to developing}, but as a \emph{design
constraint} of experimental programming infrastructure. Because of the
fragmentation of scientific programming infrastructure, where each
experimental prep is implemented with entirely different, and often
single-use software, there is no established reporting system for
automatically capturing these minute details --- but that doesn't mean
there can't be (as I wrote previously, see section 2.3 in \cite{saundersAutopilotAutomatingBehavioral2019},,and coincidentally
measured the effect of variable water droplets).

\hypertarget{the-hacker-spirit-and-celebration-of-heroism}{%
\subsubsection{The Hacker Spirit and Celebration of
Heroism}\label{the-hacker-spirit-and-celebration-of-heroism}}

Many people are attracted to systems neuroscience precisely
\emph{because} of the\ldots{} playful\ldots{} attitude we take towards
our rigs. If you want to do something, don't ask questions just break
out the \href{http://jvoigts.scripts.mit.edu/blog/review-hot-glue/}{hot
glue}, vaseline, and aluminum foil and hack at it until it does what you
want. The natural conclusion of widespread embodiment of this lovable
scamp hacker spirit is its veneration as heroism: it is a \emph{good
thing} to have done an experiment that only you are capable of doing
because that means you're the best hacker. Not unrelated is the strong
incentive to make something new rather than build on existing tools ---
you don't get publications from pull requests, and you don't get a job
without publications. The initial International Brain Laboratory
described the wily nature of neuroscientists accordingly:

\begin{leftbar}
Simply maintaining a true collaboration between 21 laboratories
accustomed to going their own way will be a major novelty in
neuroscience. \cite{abbottInternationalLaboratorySystems2017},\end{leftbar}
And yes, like the rest of the universe, perhaps the most influential
forces in this domain are inertia and entropy. Once the boulder starts
rolling down the hill of heroic idiosyncracy, tumbling along in a
semi-stable jumble\footnote{A \emph{lovely} jumble! that probably has a
  lot of good qualities, it's just a little lonely maybe :(} that
supports the experiments of a lab, retooling and standardizing that
system has to be \emph{so very cool and worth it} that it overcomes the
various, uncertain, but typically substantial costs (including the valid
emotional costs of wishing a peaceful voyage to well-loved handcrafted
tools). More than a single moment of adoption, the universe always has
room for another course of disorder, and a commitment to using communal
tools must be constantly reaffirmed. As we dream up new wild
experiments, it needs to be easier to implement them with the existing
system and integrate the labor expended in doing so back into it than it
is to patch over the problem with a quick script saved to Desktop. As
people cycle through the lab, it must be easier to learn than it is to
start from scratch.

Yes again, Mainen and colleagues:

\begin{leftbar}
Neuroscientists frequently live on the `bleeding' edge technologically,
building bespoke and customized tools. This do-it-yourself approach has
allowed innovators to get ahead of the competition, but hampered the
standardization of methods essential to making experiments efficient and
replicable.

Remarkably, it is standard practice for each lab to custom engineer all
manner of apparatus, from microscopes and electrodes to the computer
programmes for analysing data. Thousands of labs worldwide use the
calcium sensor GCaMP, for example, for imaging neural activity in vivo.
Yet neither the microscopes used for GCaMP imaging nor the algorithms
used to analyse the resulting data sets have been standardized. \cite{mainenBetterWayCrack2016},\end{leftbar}
!! make it clearer that the hacker spirit is not a \emph{bad} thing but
another \emph{design constraint} and that we should actually avoid the
paternalistic approach that says there's a ``right way'' to do science,
and instead honor, learn from, and support the diversity of our
approaches.

\hypertarget{focus-on-the-science}{%
\subsubsection{Focus on the Science}\label{focus-on-the-science}}

Completely understandably\ldots{} scientists want to focus on their
discipline rather than spending time building infrastructure. But
because infrastructure touches all of our work and very few people can
only build it in their spare time (mostly for the love of the craft) we
all have to build some of it. this is a classic collective action
problem, and scientists are not evil or selfish for wanting to do their
work.

\hypertarget{combinatorics-of-recent-technology}{%
\subsubsection{Combinatorics of Recent
Technology}\label{combinatorics-of-recent-technology}}

A lot of what I will describe here is relatively new! Some ideas are
very old, like the semantic web and wikis, but others like federated
communication and file transfer protocols are only reaching widespread
use recently. The entire universe of open source scientific hardware and
software has only sprung into its full and beautiful glory in the last
decade or so, from pandas and and jupyter to open ephys and miniscopes
and so on. Bittorrent is cool and good but IPFS allows us to think about
qualitatively different things. It's ultimately the \emph{combination}
of these recently technologies that's important, rather than any single
one of them. So in some sense it wasn't \emph{possible} to think about
the type of basic infrastructure outside the traditional lens of
centralized databases and individual experimental software packages.

Each of these three disciplinary tendencies

!! The problems are also structural, and vary depending on the size,
resources, etc. of the institution as well\ldots{} transition to next
section

\hypertarget{the-ivies-consortia-and-most-of-us}{%
\subsection{The Ivies, Consortia, and ``Most of
Us''}\label{the-ivies-consortia-and-most-of-us}}

The initial picture I painted of the state of Systems Neuroscience
describes what I, in my limited exposure to the broader field, think
might be typical for ``most of us.'' There are admirable efforts to
standardize on tools and realize ``meso-scale collaboration'' \cite{mainenBetterWayCrack2016},,and even for those that are not the
experience of infrastructure can vary dramatically by institution. To
investigate the constraints and circumstances of infrastructure
development, we can examine the different domains of contact researchers
have with organized digital technology.

\hypertarget{institutional-core-facilities}{%
\subsubsection{Institutional Core
Facilities}\label{institutional-core-facilities}}

Centralized ``core'' facilities are maybe the most typical form of
standardization and resource sharing at the level of departments and
institutions. These facilities can range from minimal to baroque
extravagance depending on institutional resources and whatever complex
web of local history brought them about.

\href{https://projectreporter.nih.gov/project_info_details.cfm?aid=9444124}{PNI
Systems Core} lists
\href{https://projectreporter.nih.gov/project_info_subprojects.cfm?aid=9444124\&icde=0}{subprojects}
echo a lot of the thoughts here, particularly around effort
duplication\footnote{Thanks a lot to the one-and-only stunning and
  brilliant Dr.~Eartha Mae Guthman for suggesting looking at the BRAIN
  initiative grants as a way of getting insight on core facilities.}:

\begin{leftbar}
Creating an Optical Instrumentation Core will address the problem that
much of the technical work required to innovate and maintain these
instruments has shifted to students and postdocs, because it has
exceeded the capacity of existing staff. This division of labor is a
problem for four reasons: (1) lab personnel often do not have sufficient
time or expertise to produce the best possible results, (2) the
diffusion of responsibility leads people to duplicate one another's
efforts, (3) researchers spend their time on technical work at the
expense of doing science, and (4) expertise can be lost as students and
postdocs move on. For all these reasons, we propose to standardize this
function across projects to improve quality control and efficiency.
Centralizing the design, construction, maintenance, and support of these
instruments will increase the efficiency and rigor of our microscopy
experiments, while freeing lab personnel to focus on designing
experiments and collecting data.
\end{leftbar}

While core facilities are an excellent way of expanding access, reducing
redundancy, and standardizing tools within an instutition, as commonly
structured they can displace work spent on those efforts outside of the
institution. Elite institutions can attract the researchers with the
technical knowledge to develop the instrumentation of the core and
infrastructure for maintain it, but this development is only
occasionally made usable by the broader public. The Princeton data
science core is an excellent example of a core facility that does makes
its software infrastructure development
\href{https://github.com/BrainCOGS}{public}\footnote{\begin{leftbar}
  Project Summary: Core 2, Data Science Working memory, the ability to
  temporarily hold multiple pieces of information in mind for
  manipulation, is central to virtually all cognitive abilities. This
  multi-component research project aims to comprehensively dissect the
  neural circuit mechanisms of this ability across multiple brain areas.
  In doing so, it will generate an extremely large quantity of data,
  from multiple types of experiments, which will then need to be
  integrated together. The Data Science Core will support the individual
  research projects in discovering relationships among behavior, neural
  activity, and neural connectivity. The Core will create a standardized
  computational pipeline and human workflow for preprocessing of
  calcium-imaging data. The pipeline will run either on local computers
  or in cloud computing services, and users will interact with it
  through a web browser. The preprocessing will incorporate existing
  image-processing algorithms, such as Constrained Nonnegative Matrix
  Factorization and convolutional networks. In addition, the Core will
  build a data science platform that stores behavior, neural activity,
  and neural connectivity in a relational database that is queried by
  the DataJoint language. Diverse analysis tools will be integrated into
  DataJoint, enabling the robust maintenance of data-processing chains.
  This data-science platform will facilitate collaborative analysis of
  datasets by multiple researchers within the project, and make the
  analyses reproducible and extensible by other researchers. We will
  develop effective methods for training and otherwise disseminating our
  computational tools and workflows. Finally, the Core will make raw
  data, derived data, and analyses available to the public upon
  publication via the data-science platform, source-code repositories,
  and web-based visualization tools. To facilitate the conduct of this
  research, the creation of software tools, and the reuse of the data by
  others after the primary research has concluded, the project will
  adopt shared data and metadata formats using the HDF5 implementation
  of the Neurodata without Borders format. Data will be made public in
  accord with the FAIR guiding principles---findndable by a DOI and/or
  URL, accessible through a RESTful web API, and interoperable and
  reusable due to DataJoint and the Neurodata Without Borders format for
  data
  https://projectreporter.nih.gov/project\_info\_description.cfm?aid=9444126\&icde=0
  \end{leftbar}}, which they should be applauded for, but also
illustrative of the problems with a core-focused infrastructure project.
For an external user, the documentation and tutorials are incomplete --
it's not clear to me how I would set this up for my institute, lab, or
data, and there are several places of hard-coded princeton-specific
values that I am unsure how exactly to adapt\footnote{Though again, this
  project is examplary, built by friends, and would be an excellent
  place to start extending towards global infrastructure.}. I would
consider this example a high-water mark, and the median openness of core
infrastructure falls far below it. I was unable to find an example of a
core facility that maintained publicly-accessible documentation on the
construction and operation of its experimental infrastructure or the
management of its facility.

\hypertarget{centralized-institutes}{%
\subsubsection{Centralized Institutes}\label{centralized-institutes}}

Outside of universities, the Allen Brain Institute is perhaps the most
impactful reflection of centralization in neuroscience. The Allen
Institute has, in an impressively short period of time, created several
transformative tools and datasets, including its well-known atlases \cite{leinGenomewideAtlasGene2007},andthe first iteration of its
\href{http://observatory.brain-map.org/}{Observatory} project which
makes a massive, high-quality calcium imaging dataset of visual cortical
activity available for public use. They also develop and maintain
software tools like their
(SDK){[}https://allensdk.readthedocs.io/en/latest/{]} and Brain Modeling
Toolkit \href{https://alleninstitute.github.io/bmtk/}{(BMTK)}, as well
as a collection of (hardware
schematics){[}https://portal.brain-map.org/explore/toolkit/hardware{]}
used in their experiments. The contribution of the Allen Institute to
basic neuroscientific infrastructure is so great that, anecdotally, and
at least in my neck of the woods, when talking about infrastructure the
default belief is ``I thought the Allen was doing that.''

Though the Allen Institute is an excellent model for scale at the level
of a single organization, its centralized, hierarchical structure cannot
(and does not attempt to) serve as the backbone for all neuroscientific
infrastructure. Performing single (or a small number of, as in its
also-admirable
\href{https://alleninstitute.org/what-we-do/brain-science/news-press/articles/three-collaborative-studies-launch-openscope-shared-observatory-neuroscience}{OpenScope
Project}) carefully controlled experiments a huge number of times is an
important means of studying constrained problems, but is complementary
with the diversity of research questions, model organisms, and methods
present in the broader neuroscientific community. Christof Koch, its
director, describes the challenge of centrally organizing a large number
of researchers:

\begin{leftbar}
Our biggest institutional challenge is organizational: assembling,
managing, enabling and motivating large teams of diverse scientists,
engineers and technicians to operate in a highly synergistic manner in
pursuit of a few basic science goals \cite{grillnerWorldwideInitiativesAdvance2016},\end{leftbar}
\begin{leftbar}
These challenges grow as the size of the team grows. Our anecdotal
evidence suggests that above a hundred members, group cohesion appears
to become weaker with the appearance of semi-autonomous cliques and
sub-groups. This may relate to the postulated limit on the number of
meaningful social interactions humans can sustain given the size of
their brain \cite{kochBigScienceTeam2016},\end{leftbar}
\hypertarget{meso-scale-collaborations}{%
\subsubsection{Meso-scale
collaborations}\label{meso-scale-collaborations}}

Given the diminishing returns to scale for centralized organizations,
many have called for smaller, ``meso-scale'' collaborations and
consortia that combine the efforts of multiple labs \cite{mainenBetterWayCrack2016},.The most successful consortium of this
kind has been the International Brain Laboratory \cite{
abbottInternationalLaboratorySystems2017, woolKnowledgeNetworksHow2020}, a group of 22 labs spread across six countries. They have been
able to realize the promise of big team neuroscience, setting a new
standard for performing reproducible experiments performed by many labs
\cite{laboratoryStandardizedReproducibleMeasurement2020},anddeveloping data management infrastructure to match \cite{laboratoryDataArchitectureLargescale2020},(seriously,don't miss
their extremely impressive
\href{https://data.internationalbrainlab.org/}{data portal}). Their
project thus serves as the benchmark for large-scale collaboration and a
model from which all similar efforts should learn from.

Critical to the IBL's success was its adoption of a flat,
non-hierarchical organizational structure, as described by Lauren E.
Wool:

\begin{leftbar}
IBL's virtual environment has grown to accommodate a diversity of
scientific activity, and is supported by a flexible, `flattened'
hierarchy that emphasizes horizontal relationships over vertical
management. {[}\ldots{]} Small teams of IBL members collaborate on
projects in Working Groups (WGs), which are defined around particular
specializations and milestones and coordinated jointly by a chair and
associate chair (typically a PI and researcher, respectively). All WG
chairs sit on the Executive Board to propagate decisions across WGs,
facilitate operational and financial support, and prepare proposals for
voting by the General Assembly, which represents all PIs. In parallel,
associate chairs convene on their own committee to share decisions,
which are then conveyed to the entire researcher community so it may
weigh in on proposals before a formal vote. The interests of PIs and
researchers intersect via staff liaisons who sit on both the Executive
Board and the Associate Chairs Committee, as well as an elected
researcher representative, who sits on the Executive Board and is a
voting member of the General Assembly. \cite{woolKnowledgeNetworksHow2020},\end{leftbar}
They should also be credited with their adoption of a form of consensus
decision-making, \href{https://sociocracy.info}{sociocracy}, rather than
a majority-vote or top-down decisionmaking structure. Consensus
decision-making systems are derived from those developed by
\href{https://rhizomenetwork.wordpress.com/2011/06/18/a-brief-history-of-consenus-decision-making/}{Quakers
and some Native American nations}, and emphasize, perhaps
unsurprisingly, the value of collective consent rather than the will of
the majority. Sociocracy specifically describes consent:

\begin{leftbar}
Consent means ``no objections.'' Giving consent does not mean unanimity,
agreement, or even endorsement. Decisions are made to guide actions. Can
we move forward if we make this decision? Consent is given in the
context of moving forward. Consent to a policy decision means you
believe that it is ``worth trying.'' Or ``I can work with it.'' Moving
forward is important for making better decisions because it provides
more information. Not moving forward until a perfect decision is found,
means operating in the blind. Information will always be limited to what
is already known.

Consent is required for all policy decisions for many reasons. The two
most important are that it ensures (1) the decision will allow all
members of the group to participate or produce without feeling
oppressed, and (2) it will be supported by everyone. Everyone is
expected to participate in the reasoning behind the decision. And no one
can be excluded. https://www.sociocracy.info/what-is-sociocracy/
\end{leftbar}

The central lesson of the IBL, in my opinion, is that governance
matters. Even if a consortium of labs were to form on an ad-hoc basis,
without a formal system to ensure contributors felt heard and empowered
to shape the project it would soon become unsustainable. Even if this
system is not perfect, with some labor still falling unequally on some
researchers, it is a promising model for future collaborative consortia.

The infrastructure developed by the IBL is impressive, but its focus on
a single experiment makes it difficult to expand and translate to
widescale use. The hardware for the IBL experimental apparatus is
exceptionally well-documented, with a
\href{https://figshare.com/articles/preprint/A_standardized_and_reproducible_method_to_measure_decision-making_in_mice_Appendix_3_IBL_protocol_for_setting_up_the_behavioral_training_rig/11634732}{complete
and detailed build guide} and
\href{https://figshare.com/articles/online_resource/A_standardized_and_reproducible_method_to_measure_decision-making_in_mice_CAD_files_for_behavior_rig/11639973}{library
of CAD parts}, but the documentation is not modularized such that it
might facilitate use in other projects, remixed, or repurposed. The
\href{https://github.com/int-brain-lab/iblrig}{experimental software} is
similarly single-purpose, a chimeric combination of Bonsai \cite{lopesBonsaiEventbasedFramework2015},and\href{https://github.com/pybpod/pybpod}{PyBpod}
\href{https://github.com/int-brain-lab/iblrig/tree/master/tasks/_iblrig_tasks_ephysChoiceWorld}{scripts}.
It unfortunately
\href{https://iblrig.readthedocs.io/en/latest/index.html}{lacks} the
API-level documentation that would facilitate use and modification by
other developers, so it is unclear to me, for example, how I would use
the experimental apparatus in a different task with perhaps slightly
different hardware, or how I would then contribute that back to the
library. The experimental software, according to the
\href{https://figshare.com/articles/preprint/A_standardized_and_reproducible_method_to_measure_decision-making_in_mice_Appendix_3_IBL_protocol_for_setting_up_the_behavioral_training_rig/11634732}{PDF
documentation}, will also not work without a connection to an
\href{https://github.com/cortex-lab/alyx}{alyx} database. While alyx was
intended for use outside the IBL, it still has
\href{https://github.com/cortex-lab/alyx/blob/07f481f6bbde668b81ad2634f4c42df4d6a74e44/alyx/data/management/commands/files.py\#L188}{IBL-specific}
and
\href{https://github.com/cortex-lab/alyx/blob/07f481f6bbde668b81ad2634f4c42df4d6a74e44/alyx/data/fixtures/data.datasettype.json\#L29}{task-specific}
values in its source-code, and makes community development difficult
with a similar \href{https://alyx.readthedocs.io/en/latest/}{lack} of
API-level documentation and requirement that users edit the library
itself, rather than temporary user files, in order to use it outside the
IBL.

My intention is not to denigrate the excellent tools built by the IBL,
nor their inspiring realization of meso-scale collaboration, but to
illustrate a problem that I see as an extension of that discussed in the
context of core facilities --- designing infrastructure for one task, or
one group in particular makes it much less likely to be portable to
other tasks and groups.

It is also unclear how replicable these consortia are, and whether they
challenge, rather than reinforce technical inequity in science.
Participating in consortia systems like the IBL requires that labs have
additional funding for labor hours spent on work for the consortium, and
in the case of graduate students and postdocs, that time can conflict
with work on their degrees or personal research which are still far more
potent instruments of ``remaining employed in science'' than
collaboration. In the case that only the most well-funded labs and
institutions realize the benefits of big team science without explicit
consideration given to scientific equity, mesoscale collaborations could
have the unintended consequence of magnifying the skewed distribution of
access to technical expertise and instrumentation.

\hypertarget{platforms-scientific-saas}{%
\subsubsection{Platforms \& Scientific
SaaS}\label{platforms-scientific-saas}}

!! (todo more explicit about the market dynamics right now driving
people to make lots of small scientific startups and then eg get bought
by sigma. then about the problem of the conflicting incentives towards
interoperability -- the platforms, by design, must generate some
inefficiency that makes them stay in business, eg. like pharma companies
researching how to treat rather than cure diseases.)

\hypertarget{open-source-scientific-software}{%
\subsubsection{Open-Source Scientific
Software}\label{open-source-scientific-software}}

!! (section on funding mechanisms for open source, the drive to publish
separable packages rather than PR existing tools becasue of publication
incentives, the lack of formal organizing systemsfor multipackage
development)

\hypertarget{the-rest-of-us}{%
\subsubsection{The rest of us\ldots{}}\label{the-rest-of-us}}

Outside of ivies with rich core facilities, institutes like the Allen,
or nascent multi-lab consortia, the situation errs closer to the dire
picture of fragmentation I painted above. In addition to the homebrew
stuff, there is an ocean of open-source software and hardware that keeps
us afloat. There are far too many projects to name here\footnote{That we
  love!!!! Pay developers!!!!!}, each covering some subset of
experimenters needs, only rarely integrated with one another, and so to
some degree the task of many scientific programmers is to search out the
latest packages and quilt them into our patchwork local infrastructure.
Anything else comes down to whatever we can afford with remaining grant
money, scrape together from local knowledge, methods sections, begging,
borrowing, and (hopefully not too much) stealing from neighboring labs.

A third option from the standardization offered by centralization and
the blooming, buzzing, beautiful chaos of disconnected open-source
development is that of decentralized systems. Rather than systems of
geographically decentralized \emph{people or lab-sites,} what must be
decentralized is the \emph{infrastucture itself:} we need to build the
means by which the ``rest of us'' can mutually benefit by capturing and
making use of each other's knowledge and labor.

\hypertarget{a-vision-of-distributed-scientific-infrastructure}{%
\section{A Vision of Distributed Scientific
Infrastructure}\label{a-vision-of-distributed-scientific-infrastructure}}

The distributed infrastructure I will describe here is related to
previous notions of ``grass-roots'' science \cite{mainenBetterWayCrack2016},,and my intention is to provide a more
prescriptive scaffolding for its design and potential implementation as
a way of painting a picture of what science could be like.

Throughout this section, when I am referring to any particular piece of
software I want to be clear that I don't intend to be dogmatically
advocating that software \emph{in particular}, but software \emph{like
it} that \emph{shares its qualities} --- no snake oil is sold in this
document. Since this is a design document, I will also be saying we
\emph{should} do a lot of things --- think of that as ``to fulfill this
system, we should do this,'' rather than ``everyone should do this even
if they disagree with the fundament of my argument.'' Similarly, when I
describe limitations of existing tools, without exception I am
describing a tool or platform I love, have learned from, and think is
valuable --- learning from something can mean drawing respectful
contrast!

At all points, I assume that the particular tool has a \emph{well
designed UI/UX} such that it is relatively simple to use and understand
--- if it takes a college degree to turn the water on, then it ain't
infrastructure. All the things I describe here either already exist or
are extensions of things that exist, so good design may require
improvements but is in all cases possible. Practicality matters:
infrastructure will only work of it's widely adopted, and it will only
be widely adopted if it is easier and more rewarding to use than the
costs of transition.

I won't attempt a derivation of a definition of decentralized systems
from base principles here, but a concrete example of one is very close
to home: the internet (or, specifically, the Internet Protocol, or IP).
The history of the internet is, at the time of writing, still very near
at hand, and much of its design philosophy has been carefully
articulated by the engineers and designers that created it. A small
selection of these principles hint at what might be required of
distributed infrastructure for neuroscience, in no particular order:

\begin{itemize}

\item
  \textbf{Integrate with what exists} - At its advent, several different
  institutions and universities had already developed existing network
  infrastructures, and so the ``top level goal'' of the Internet
  Protocol was to ``develop an effective technique for multiplex
  utilization of existing interconnected networks,'' and ``come to grips
  with the problem of integrating a number of separately administered
  entities into a common utility'' \cite{clarkDesignPhilosophyDARPA1988},.As a result, IP was developed as
  a `common language' that could be implemented on any hardware, and
  upon which other, more complex tools could be built. This is also a
  cultural practice: when the system doesn't meet some need, one should
  try to extend it rather than building a new, separate system --- and
  if a new system is needed, it should be interoperable with those that
  exist.
\item
  \textbf{Empower the end-user} - Becauase IP was initially developed as
  a military technology by DARPA, a primary design constraint was
  survivability in the face of failure. The model adopted by internet
  architects was to move as much functionality from the network itself
  to the end-users of the network --- rather than the network itself
  guaranteeing a packet is transmitted, the sending computer will do so
  by requiring a response from the recipient. For infrastructure, we
  should make tools that don't require a central team of developers to
  maintain, a central server-farm to host data, or a small group of
  people to govern. Whenever possible, data, software, and hardware
  should be self-describing, so one needs minimal additional tools or
  resources to understand and use it.
\item
  \textbf{Modularity is Flexibility} - Building each component once, and
  once only requires that it ``knows'' about as few other parts of the
  system as possible. Once a component is built well, it can be reused
  and repurposed in contexts not imagined in its original design.
  Modularity is also critical for large-scale use: partial adoption
  partially captures development labor. Allowing users to gradually
  incorporate the pieces of a system into their existing infrastructure
  also lowers the barriers of high transitional costs to eventual
  complete adoption. Modularity applies at all scales -- the individual
  components of eg. an analytical framework should be independent from
  one another, but so too should the form of the analytical framework
  from the means of sharing the data it analyzes. A reciprocal principle
  to modularity is ``the test of independent invention'', or ``If
  someone else had already invented your system, would theirs work with
  yours?'' \cite{berners-leePrinciplesDesign1998},.In other
  words, in addition to the system itself being modular, it should also
  be designed so there is some sensible means for it to be integrated
  into some yet-unspecified larger project. The machine needs to have
  knobs.
\item
  \textbf{Embrace Heterogeneity} - Distributed systems need to
  anticipate unanticipated uses. Rather than a prescribed set of
  supported hardware, affordance needs to be made such that there is a
  clear way to extend the system to incorporate new function \cite{carpenterRFC1958Architectural1996},.\item
  \textbf{Scalability is The Metric} - The system needs to have minimal
  barriers to use such that it can be deployed by as many users as
  possible --- scale is not just a design principle, but an independent
  objective and means of valuation for distributed systems. Logic or
  functionality that can only be used by a specific set of users breaks
  the system. Hand in hand with embracing heterogeneity, infrastructure
  needs to be able to be adopted by users with a minimal set of
  assumptions about their resources, organization, or expertise.
\end{itemize}

With these principles in mind, and drawing from other knowledge
communities solving similar problems: internet infrastructure,
library/information science, peer-to-peer networks, and radical
community organizers, I conceptualize a system of distributed
infrastructure for systems neuroscience as four objectives:
\protect\hyperlink{shared-data}{\textbf{shared data}},
\protect\hyperlink{shared-tools}{\textbf{shared tools}},
\protect\hyperlink{shared-knowledge}{\textbf{shared knowledge}}, and
\protect\hyperlink{shared-governance}{\textbf{shared governance}}.

\hypertarget{shared-data}{%
\subsection{Shared Data}\label{shared-data}}

\hypertarget{common-format}{%
\subsubsection{Common Format}\label{common-format}}

Neuroscientific data should be stored in a single, common format. Given
the absence of notable competitors and existing partial standardization,
we should adopt \href{rubelNWBAccessibleData2019a}{Neurodata Without
Borders:N}. I don't expect a lot of controvery here\footnote{but I am
  also almost always wrong when I think this.}. Individual labs writing
functions for converting their data to NWB is a comparatively simple,
concrete first step that is a prerequitite for the remainder of the
system. It could even be fun, we could think of it like a big years-long
slumber party where we all learn one dance routine.

!! change this section to something about the importance of
relationships between data formats, simplify that way. introduce the
concept for later where we talk about federation and later in rdf.
another framing might be provenance https://www.w3.org/TR/prov-overview/

Standardization does \emph{not} mean that it is the \emph{only} format
that is used --- there are legitimate applications for keeping data,
even temporarily, in intermediate formats. Standardization, in this
case, means that the data has some trivial conversion to and from NWB,
so for example some experimental tool could implement its own data model
as long as it could be exported to NWB.

Relatedly, the NWB API should be extended to include conversions between
prior and subsequent versions of the standard: when the standard is
changed, there should also be a function that converts the previous to
the new version and vice versa. Once data is in NWB, it would then be
trivial to maintain compatibility while allowing the standard to evolve
as needed.

If such a conversion function was implemented such that it was easy to
extend, then it would also allow researchers to make local modifications
to the standard to suit their needs, while retaining standardization
with the root format. NWB files would then effectively be version
controlled, and innovation on the standard could be integrated into the
root standard and made available to all existing data. If
compatibility-preserving extension of the protocol was possible, then
the temptation to create accessory file and directory storage to contain
``undefinable'' data is minimized and it becomes possible to additional
stipulate that doing so should be avoided.

Wide adoption of a standardized data format is not, in my opinion, the
end goal in itself. Instead I see it as a point of standardization on
the way to a more generalized, interlinked system of linked schema,
articulated further in the following sections and in the discussion of
\protect\hyperlink{shared-knowledge}{shared knowledge}.

\hypertarget{peer-to-peer-data-sharing-platform}{%
\subsubsection{Peer-to-peer data sharing
platform}\label{peer-to-peer-data-sharing-platform}}

We should develop a platform for sharing all neuroscientific data. There
are, of course \href{https://www.dandiarchive.org/}{many}
\href{https://openneuro.org/}{existing}
\href{https://www.brainminds.riken.jp/}{databases}
\href{https://biccn.org/}{for} scientific data, ranging from
domain-general like \href{https://figshare.com/}{figshare} and
\href{https://zenodo.org/}{zenodo} to the most laser-focused
subdiscipline-specific. For all these databases, their centralization is
a fundamental constraint to adoption and growth. We can learn from two
knowledge communities with decades of domain-specific knowledge in
resiliently storing and sharing massive quantities of extremely
heterogeneous and metadata-rich information: internet pirates and
information scientists. We should develop a peer-to-peer, semantically
annotated data sharing platform.

Centralized servers are fundamentally constrained by their storage
capacity and bandwidth, both of which cost money. In order to be free,
database maintainers need to constantly raise money from donations or
grants\footnote{granting agencies seem to love funding new databases,
  idk.} in order to pay for both. Funding can never be infinite, and so
inevitably there must be some limit on the amount of data that someone
can upload and the speed at which it can serve files\footnote{As I am
  writing this, I am getting a (very unscientific) maximum speed of
  5MB/s on the \href{https://osf.io}{Open Science Framework}}. Even if
the database is carefully backed up, it serves as a single point of
infrastructural failure, where if the project lapses then at worst data
will be irreversibly lost, and at best a lot of labor needs to be
expended to exfiltrate, reformat, and rehost the data. The same is true
of local, institutional-level servers and related database platforms,
with the additional problem of skewed funding allocation making them
unaffordable for many researchers.

Peer to peer systems have none of these problems, are inexpensive to
maintain, and increase, rather than decrease, in performance the more
people use them. In order to proceed with the rest of this section we
need to give a brief description of a peer to peer networking protocol:
\href{https://en.wikipedia.org/wiki/BitTorrent}{Bittorrent}. If you are
already familiar with the basics of Bittorrent, you can safely collapse
and skip the next section. Note that I am just using Bittorrent as an
example, contemporary P2P systems have made substantial improvements on
Bittorrent\footnote{peer to peer systems are, maybe predictably, a whole
  academic subdiscipline. See \cite{shenHandbookPeertoPeerNetworking2010},forreference.}, explained
after the interlude.

\hypertarget{bittorrent-interlude}{%
\paragraph{Bittorrent Interlude}\label{bittorrent-interlude}}

The above illustration of an oversimplified peer-to-peer network by
itself has the capability of providing a more robust, resilient
infrastructure for the massive datasets in neuroscience. Entry costs are
low, any existing server infrastructure present in labs, institutes,
etc. can use a peer to peer system. Peer-to-peer networks also
theoretically allow the maximum bandwidth of an entire networking system
to be used, rather than the maximum bandwidth of a single connection.

Peer to peer systems and server/client are not, in fact, mutually
exclusive: peer to peer systems should \emph{always} be \emph{at least}
as fast and have \emph{at least} as much storage as the alternative
server/client model that would have otherwise been implemented. It is
possible for a server to play the role of an ``obligate
peer\footnote{or, in the parlance of bittorrent, a
  \href{https://en.wikipedia.org/wiki/BitTorrent\#Web_seeding}{web seed}}''
in a network where it always automatically downloads everything that is
uploaded, so in that case the benefit of the peer-to-peer system is
strictly additive. Since there is nothing special about the obligate
peer (let's just call it the the server, it still is) in the swarm, it
is possible for arbitrarily many server farms to be combined to expand
the redundancy and speed of the system. The obligate peer arrangement
prevents the biggest problem of peer-to-peer networks where a file can
become unavailable if everyone who has it stops uploading it. In doing
so it can also serve as a load balancer in the network, where
less-common datasets receive more of the server's bandwidth than common
ones.

There are many improvements and variations on peer to peer technology
that would make it more suitable for scientists. A scientific
peer-to-peer system needs to be capable of version control across
iterations of a dataset, to be able to control permissions for datasets,
to be able to serve partial datasets (eg. a NWB dataset is a single
file, but it should be possible to download the behavior data without
downloading the raw 2-photon data), etc. The network can be made more
robust by incorporating automatic replication, where users of the
network volunteer to share part of their storage space which is then
automatically filled with (encrypted) shards of data from the rest of
the network (see, for one example among many, Freenet \cite{clarkeFreenetDistributedAnonymous2001},).This scheme ensures that
even if the last peer that is explicitly hosting a particular dataset
drops out, the dataset will always persist distributed through the
network, provided enough shared storage is present.

These scattered suggestions are meant to illustrate the flexibility and
variability from the simplest peer-to-peer architecture, and
fine-grained details of their implementation and an enumeration of the
possible systems are far outside the scope of this paper. I will return
to consider the design requirements of a scientific peer-to-peer network
after discussing community overlays, the second half of the peer-to-peer
story.

!! \cite{langilleBioTorrentsFileSharing2010},DANDIis in on the
p2p system, as is kachery-p2p

\hypertarget{archives-need-communities}{%
\paragraph{Archives Need Communities}\label{archives-need-communities}}

An underappreciated element of the torrent system is the effect of the
separation between the data transfer protocol and the `discovery' part
of the system --- or ``overlay'' --- on the community structure of
torrent trackers. Many peer to peer networks like KaZaA or the
gnutella-based Limewire had searching for files integrated into the
transfer interface. The need for torrent trackers to share .torrent
files spawned a massive community of private torrent trackers that for
decades have been iterating on cultures of archival, experimenting with
different community structures and incentives that encourage people to
share and annotate some of the world's largest, most organized
libraries. One of these private trackers was the site of one of the
largest informational tragedies of the past decade: what.cd\footnote{for
  a detailed description of the site and community, see Ian Dunham's
  dissertation \cite{dunhamWhatCDLegacy2018},}
What.cd was a bittorrent tracker that was arguably the largest
collection of music that has ever existed. At the time of its
destruction in 2016, it was host to just over one million unique
releases, and approximately 3.5 million torrents\footnote{Though spotify
  now boasts its library having 50 million tracks, back of the envelope
  calculations relating number of releases to number of tracks are
  fraught, given the long tail of track numbers on albums like classical
  music anthologies with several hundred tracks on a single ``release.''}
\cite{dunhamWhatCDLegacy2018},.Every torrent was organized in a
meticulous system of metadata communally curated by its roughly 200,000
global users. The collection was built by people who cared deeply about
music, rather than commercial collections provided by record labels
notorious for ceasing distribution of recordings that are not
commercially viable --- or just losing them in a fire \cite{rosenDayMusicBurned2019},\footnote{\begin{leftbar}  "Among the incinerated Decca masters were recordings by titanic
  figures in American music: Louis Armstrong, Duke Ellington, Al Jolson,
  Bing Crosby, Ella Fitzgerald, Judy Garland. The tape masters for
  Billie Holiday's Decca catalog were most likely lost in total. The
  Decca masters also included recordings by such greats as Louis Jordan
  and His Tympany Five and Patsy Cline.

  The fire most likely claimed most of Chuck Berry's Chess masters and
  multitrack masters, a body of work that constitutes Berry's greatest
  recordings. The destroyed Chess masters encompassed nearly everything
  else recorded for the label and its subsidiaries, including most of
  the Chess output of Muddy Waters, Howlin' Wolf, Willie Dixon, Bo
  Diddley, Etta James, John Lee Hooker, Buddy Guy and Little Walter.
  Also very likely lost were master tapes of the first commercially
  released material by Aretha Franklin, recorded when she was a young
  teenager performing in the church services of her father, the
  Rev.~C.L. Franklin, who made dozens of albums for Chess and its
  sublabels.

  Virtually all of Buddy Holly's masters were lost in the fire. Most of
  John Coltrane's Impulse masters were lost, as were masters for
  treasured Impulse releases by Ellington, Count Basie, Coleman Hawkins,
  Dizzy Gillespie, Max Roach, Art Blakey, Sonny Rollins, Charles Mingus,
  Ornette Coleman, Alice Coltrane, Sun Ra, Albert Ayler, Pharoah Sanders
  and other jazz greats. Also apparently destroyed were the masters for
  dozens of canonical hit singles, including Bill Haley and His Comets'
  ``Rock Around the Clock,'' Jackie Brenston and His Delta Cats'
  ``Rocket 88,'' Bo Diddley's ``Bo Diddley/I'm A Man,'' Etta James's
  ``At Last,'' the Kingsmen's ``Louie Louie'' and the

  The list of destroyed single and album masters takes in titles by
  dozens of legendary artists, a genre-spanning who's who of 20th- and
  21st-century popular music. It includes recordings by Benny Goodman,
  Cab Calloway, the Andrews Sisters, the Ink Spots, the Mills Brothers,
  Lionel Hampton, Ray Charles, Sister Rosetta Tharpe, Clara Ward, Sammy
  Davis Jr., Les Paul, Fats Domino, Big Mama Thornton, Burl Ives, the
  Weavers, Kitty Wells, Ernest Tubb, Lefty Frizzell, Loretta Lynn,
  George Jones, Merle Haggard, Bobby (Blue) Bland, B.B. King, Ike
  Turner, the Four Tops, Quincy Jones, Burt Bacharach, Joan Baez, Neil
  Diamond, Sonny and Cher, the Mamas and the Papas, Joni Mitchell,
  Captain Beefheart, Cat Stevens, the Carpenters, Gladys Knight and the
  Pips, Al Green, the Flying Burrito Brothers, Elton John, Lynyrd
  Skynyrd, Eric Clapton, Jimmy Buffett, the Eagles, Don Henley,
  Aerosmith, Steely Dan, Iggy Pop, Rufus and Chaka Khan, Barry White,
  Patti LaBelle, Yoko Ono, Tom Petty and the Heartbreakers, the Police,
  Sting, George Strait, Steve Earle, R.E.M., Janet Jackson, Eric B. and
  Rakim, New Edition, Bobby Brown, Guns N' Roses, Queen Latifah, Mary J.
  Blige, Sonic Youth, No Doubt, Nine Inch Nails, Snoop Dogg, Nirvana,
  Soundgarden, Hole, Beck, Sheryl Crow, Tupac Shakur, Eminem, 50 Cent
  and the Roots.

  Then there are masters for largely forgotten artists that were stored
  in the vault: tens of thousands of gospel, blues, jazz, country, soul,
  disco, pop, easy listening, classical, comedy and spoken-word records
  that may now exist only as written entries in discographies." \cite{rosenDayMusicBurned2019},\end{leftbar}}.Users would spend large amounts of money to find and
digitize extremely rare recordings, many of which were unavailable
anywhere else and are now unavailable anywhere, period. One former user
describes one example:

\begin{leftbar}
``I did sound design for a show about Ceaușescu's Romania, and was able
to pull together all of this 70s dissident prog-rock and stuff that has
never been released on CD, let alone outside of Romania'' \cite{sonnadEulogyWhatCd2016},\end{leftbar}
\includegraphics[width=\linewidth]{/Users/jonny/git/sneakers-the-rat.github.io/_preblog/assets/images/kanye-what.png} \emph{The what.cd
artist page for Kanye West (taken from
\href{https://qz.com/840661/what-cd-is-gone-a-eulogy-for-the-greatest-music-collection-in-the-world/}{here}
in the style of pirates, without permission). For the album ``Yeezus,''
there are ten torrents, grouped by each time the album was released on
CD and Web, and in multiple different qualities and formats (.flac,
.mp3). Along the top is a list of the macro-level groups, where what is
in view is the ``albums'' section, there are also sections for bootleg
recordings, remixes, live albums, etc.}

What makes hundreds of thousands of people spend massive amounts of time
for literally zero (or negative) compensation to curate a collection of
metadata? Though I am not so naive as to think it is the sole cause, I
argue that it is the community structure of the what.cd ``overlay.''
Though much of the community structure I describe here would need to be
adapted to the needs of a scientific archive, they are an important
illustration of a system that aligns the incentives of its users and
provides the tools for them to perform the distributed work of curation.

What.cd was a ``private'' bittorrent tracker, where unlike public
trackers that anyone can access, membership was strictly limited to
those who were personally invited or to those who passed an interview.
Invites were extremely rare, and the interview process was demanding to
the point where
\href{https://opentrackers.org/whatinterviewprep.com/index.html}{entire
guides} were written to prepare for them. When I interviewed in 2009, I
had to find my way onto an obscure IRC server, wait in a lobby all day
until a volunteer moderator could get to me, and was then grilled on the
arcana of digital music formats, spectral analysis\footnote{The average
  what.cd user was, as a result, on par with many of the auditory
  neuroscientists I know in their ability to read a spectrogram.}, the
ethics of piracy, and so on for half an hour. Getting a question wrong
was an instant failure and you were banned from the server for 48 hours.
A single user was only allowed one account per lifetime, so between that
policy and the extremely high barriers to entries, even anonymous users
were strongly incentivized to follow
\href{https://opentrackers.org/whatinterviewprep.com/prepare-for-the-interview/what-cd-rules/index.html}{the
sophisticated, exacting rules for contributing}.

The what.cd incentive system was based on a required ratio of data
uploaded vs.~data downloaded. Peer to peer systems need to overcome a
free-rider problem where users might download a torrent (``leeching'')
and turn their computer off, rather than leaving their connection open
to share it to others (or, ``seeding''). In order to download additional
music, then, one would have to upload more. Since downloading is highly
restricted, and everyone is trying to upload as much as they can,
torrents had a large number of ``seeders,'' and even rare recordings
would be sustained for years, a pattern common to private trackers \cite{liuUnderstandingImprovingRatio2010},.
The high seeder/leecher ratio made it so it was extremely difficult to
acquire upload credit, so users were additionally incentivized to find
and upload new recordings to the system. What.cd implemented a
``bounty'' system, where users with a large amount of excess upload
credit would be able to offer some of it to whoever was able to upload
the album they wanted. To ``prime the pump'' and keep the economy
moving, highlight artists in an album of the week, or direct users to
preserve rare recordings, moderators would also use a ``freeleech''
system, where users would be able to download a specified set of
torrents without it counting against their download quantity.

The other half of what.cd was its community infrastructure, its forums,
comment sections, and moderation systems. The forum was home to roiling
debates that lasted years about the structure of some tagging schema,
whether one genre was just another with a different name, and so on. The
structure of the community was an object of constant, public
negotiation, and over time the metadata system evolved to be able to
support a library of the entirety of human music output\footnote{Though
  music metadata might seem like a trivial problem (just look at the
  fields in an MP3 header), the number of edge cases are profound. How
  would you categorize an early Madlib casette mixtape remastered and
  uploaded to his website where he is mumbling to himself while
  recording some live show performed by multiple artists, but on the
  b-side is one of his Beat Konducta collections that mix together
  studio recordings from a collection of other artists? Who is the
  artist? How would you even identify the unnamed artists in the live
  show? Is that a compilation or a bootleg? Is it a cassette rip, a
  remaster, or a web release?}, and the rules and incentive structures
were made to align with building it. To support the good operation of
the site, the forums were also home to a huge amount of technical
knowledge, like guides on how to make a perfect upload, that eased new
users into being able to use the system.

A critical problem in maintaining coherent databases is correcting
metadata errors and departures from schemas. Finding errors was
rewarded. Users were able to discuss and ask questions of the uploader
in a comment section below each upload, which would allow ``polite''
resolution of low-level errors like typos. More serious problems could
be reported to the moderation team, which caused the upload to be
visibly marked as under review, and the report could then be discussed
either in the comment sections or the forum. If the moderation team
affirmed your report, they would usually kick back a few gigabytes of
upload credit depending on the severity. Rather than being a messy
hodgepodge of fake, low-quality uploads, what.cd was always teetering
just shy of perfection.

These structural considerations do not capture the most elusive but
indisputably important features of what.cd's community infrastructure:
\emph{the sense of commmunity}. The What.cd forums were the center of
many user's relationships to music. Threads about all the finest scales
of music nichery could last for years: it was a rare place people who
probably cared a little bit too much about music could talk to people
with the same condition. What made it more satisfying than other music
forums was that no matter what music you were talking about, everyone
else in the conversation would always have access to it if they wanted
to hear it. Independent musicians released albums in the
supportive\footnote{Mostly. You know how the internet goes\ldots{}}
Vanity House section, and people from around the world came to hold the
one true album that only they knew about high aloft like a divine
tablet. Beyond any structural incentives, people spent so much time
building and maintaining what.cd because it became a source of community
and a sink of personal investment. I'll tease a brief recurring dream
I've been having recently of something similar existing for scientists:
a place where we can discuss our experiments in the same place that they
live, being able to link to, embed, and compare data in the kind of
longform, thoughtful way that currently has no place outside of papers
in scientific culture.

This system created not only a huge, well-annotated library, but its
distributed nature made it resilient. When it was shut down, a series of
successors popped up using the open source tools
\href{https://github.com/WhatCD/Gazelle}{Gazelle} and
\href{https://github.com/WhatCD/Ocelot}{Ocelot} that what.cd developers
built. Within two weeks, one successor site had recovered and reindexed
200,000 of its torrents resubmitted by former users \cite{vandersarWhatCdDead2016},Manyfeatures of what.cd's structure are undesirable for scientific
infrastructure, but they demonstrate that a robust archive is not only a
matter of building a database with some frontend, but by building a
community \cite{brossCommunityCollaborationContribution2013},.
In contrast to what.cd, a scientific peer-to-peer system's incentives
need to (among others)\ldots{} * Have extremely \textbf{low barriers} to
entry * \textbf{Not use downloading as its ``cost''} --- users
downloading and analyzing huge amounts of data is \emph{good} and what
we \emph{want}. Other systems of incentivizing uploading and curation
have been developed by other trackers. One example is a ratioless
system, where users are required to remain seeding data they download,
either forever or for a specific amount of time. If a user keeps 100\%
of their downloads seeded, they have zero ratio requirements, which then
scale back up if the user stops sharing. * \textbf{Be Resource and
Clout-Agnostic} --- researchers with access to huge server farms or
large professional networks should not be favored by the system, which
is intended to \emph{reduce} inequity rather than \emph{reflect} it. We
\emph{don't} want to make some leaderboard system, but find ways to
incentivize thoughtful, generative archivalism.

Rather than being prescriptive about one community structure, however,
what allowed the community structure of private bittorent trackers to
develop and experiment with many different types of systems in a shared
framework. Our goal should \emph{not} be to make yet another single,
subdisciplinary-specific database. We should learn from the
meta-structure of the torrent system and take advantage of separating a
protocol from its overlay and make a \emph{federated} peer-to-peer
system.

!! some forums already exist: https://neurostars.org/

\hypertarget{federated-systems}{%
\paragraph{Federated Systems}\label{federated-systems}}

!! compare to datalad -
http://handbook.datalad.org/en/latest/usecases/datastorage\_for\_institutions.html

There is no shortage of databases for scientific data, what limits their
use is their fragmentation. Each subdiscipline having a separate
database makes combining information from across even extremely similar
subdisciplines combinatorically complex and laborious. It also makes
finding the correct database for a given dataset often a matter of
having prior knowledge or wild luck.

Even if one were to have the rare omniscience of a full and masterful
understanding of the database landscape, researchers using them are in a
bind between domain-generality and specificity. General-purpose
databases like figshare\footnote{No shade to Figshare, which, among
  others, paved the way for open data and are a massively useful thing
  to have in society.} are essentially public, versioned, folders with a
DOI, but the metadata for organizing multiple datasets together are
relatively sparse attributes like keywords, links to the DOI of the
paper, authors, etc. Domain-specific databases are more likely to have a
metadata structure that fully describes and is compatible with a
researcher's particular data, as well as visualization, summarization,
and aggregation features purpose-built for that data. The researcher can
either spend the extra time uploading to multiple databases, avoid
contributing to data fragmentation by using a general-purpose database,
or risk obscurity by using a domain-specific one.

Any single database system can only be perfectly-fit to a small slice of
the scientific population, so the solution is neither the creation of
``the one true perfect database'' nor is it creating additional,
increasingly specific databases. Matthew J Bietz and Charlotte P Lee
articulate this tension better than I can in their ethnography of
metagenomics databases:

\begin{leftbar}
"Participants describe the individual sequence database systems as if
they were shadows, poor representations of a widely-agreed-upon ideal.
We find, however, that by looking across the landscape of databases, a
different picture emerges. Instead, each decision about the
implementation of a particular database system plants a stake for a
community boundary. The databases are not so much imperfect copies of an
ideal as they are arguments about what the ideal Database should be.
{[}\ldots{]}

When the microbial ecology project adopted the database system from the
traditional genomic ``gene finders,'' they expected the database to be a
boundary object. They knew they would have to customize it to some
extent, but thought it would be able to ``travel across borders and
maintain some sort of constant identity''. In the end, however, the
system was so tailored to a specific set of research questions that the
collection of data, the set of tools, and even the social organization
of the project had to be significantly changed. New analysis tools were
developed and old tools were discarded. Not only was the database ported
to a different technology, the data itself was significantly
restructured to fit the new tools and approaches. While the database
development projects had begun by working together, in the end they were
unable to collaborate. The system that was supposed to tie these groups
together could not be shielded from the controversies that formed the
boundaries between the communities of practice." \cite{bietzCollaborationMetagenomicsSequence2009},\end{leftbar}
Here again neuroscientists could learn from other knowledge communities
trying to solve problems with parallel structure, in this case by
considering \textbf{federated} information systems. Federated systems
consist of \emph{distributed}, \emph{heterogeneous}, and
\emph{autonomous} agents that implement some minimal agreed-upon
standards for mutual communication and (co-)operation. A practical
example of a federated system is email: you can choose from a variety of
email services, each of which could have a wholly different set of
features and design, but you can still send anyone\footnote{dont @ me
  about html vs plain text messages, providers with varying degrees of
  message authentication that get bounced by others, ya know what i
  mean.} an email. More recent examples are the
\href{https://matrix.org/}{Matrix messaging protocol} and the
\href{https://en.wikipedia.org/wiki/Fediverse}{``Fediverse''} built on
W3C's
\href{https://www.w3.org/TR/2018/REC-activitypub-20180123/}{ActivityPub}
protocol \cite{Webber:18:A},forsocial networks. Users in
ActivityPub networks, rather than joining a single service as one would
with traditional commercial social media networks, join individual
servers (or can create their own). Each server chooses its own software
that implements the ActivityPub standard, and is free to set its own
rules, privileges, and whether or not it wants to be able to send and
receive messages from other servers.

For the sake of this paper, I'll focus on federated databases. Federated
databases\footnote{though there are subtleties to the terminology, with
  related terms like ``multidatabase,'' ``data integration,'' and ``data
  lake'' composing subtle shades of a shared idea. I will use federated
  databases as a single term that encompasses these multiple ideas here,
  for the sake of constraining the scope of the paper.} were proposed in
the early 1980's \cite{heimbignerFederatedArchitectureInformation1985},andhave been
developed and refined in the decades since as an alternative to
centralization or non-integration \cite{
litwinInteroperabilityMultipleAutonomous1990,
kashyapSemanticSchematicSimilarities1996,
hullManagingSemanticHeterogeneity1997} -- and their application to
the dispersion of scientific data in local filesystems is not new \cite{busseFederatedInformationSystems1999},.
Amit Sheth and James Larson, in their reference description of federated
database systems, describe the \emph{design autonomy} as one critical
dimension that characterizes them:

\begin{leftbar}
Design autonomy refers to the ability of a component DBS to choose its
own design with respect to any matter, including

\begin{enumerate}
\def\labelenumi{(\alph{enumi})}

\item
  The \textbf{data} being managed (i.e., the Universe of Discourse),
\item
  The \textbf{representation} (data model, query language) and the
  \textbf{naming} of the data elements,
\item
  The conceptualization or \textbf{semantic interpretation} of the data
  (which greatly contributes to the problem of semantic heterogeneity),
\item
  \textbf{Constraints} (e.g., semantic integrity constraints and the
  serializability criteria) used to manage the data,
\item
  The \textbf{functionality} of the system (i.e., the operations
  supported by system),
\item
  The \textbf{association and sharing with other systems}, and
\item
  The \textbf{implementation} (e.g., record and file structures,
  concurrency control algorithms).
\end{enumerate}
\end{leftbar}

Susanne Busse and colleagues add an additional dimension of
\textbf{evolvability}: ``Following''natural" tendencies, autonomous
components will inevitably develop heterogeneous structures. It is the
task of the federation layer to cope with the different types of
heterogeneity." \cite{busseFederatedInformationSystems1999},.In
the case of federated database systems, the federation layer provides a
uniform way to mediate differences in schemas and formats between
individual databases in the system. To share data between subdisciplines
and fields we need to be able to perform some \emph{mapping} between the
different data formats and standards that they use: we need some way of
translating the neuroscientist's \texttt{GENOTYPE} to the geneticists
\texttt{GENETIC\_SEQUENCE}. I will be purposefully vague about the means
of implementing these mappings until we reach the
\protect\hyperlink{shared-knowledge}{shared knowledge} section, but
first we need a brief practical example of how a system like this might
work.

Say I'm a neuroscientist who just collected a dataset that consists of a
few electrophysiological recordings from a cluster of Consciousness
Cells in some obscure midbrain nucleus, and then sectioned the brain and
imaged their positions. I deposit my dataset on my local in-lab server,
which I have set up to federate with the fancy new Neurophysiologist's
Extravagant, Undying, Repository of Open data (NEUROd). All servers in
this federation are required to have their data in the standardized NWB
format, and since mine already is (go me!) my server announces to the
others that we have some new data available! Some enterprising group of
neuroscientific programmers has built a website that allows its users to
search, browse, and do all the fancy visualization of data they would
expect from a modern database, so I go and see how my new dataset has
changed some standard aggregated analysis of all the Concscious Cells
from all the other labs participating in the federation. Hang on, I say,
a question mark appearing over my head like a cartoon caricature of a
curious scientist -- I wonder if these Consciousness Cells are in the
same place in the evolutionary neighbors of my model organism!? I then
run a query for all datasets that have positional data for Consciousness
Cells. NEUROd has chosen to federate with the Evolutionary Volitional
data sharing Operation (EVO), a federation of evolutionary biologists,
some of whom study the origins of Consciousness Cells. They have their
data in their own evolutionary biologist-specific format, but since
there is some mapping between fields in the NWB standard and theirs,
that's no problem. My search then returns data from not only all the
other neuroscientists in NEUROd, then, but also matching data from EVO
--- and my cross-disciplinary question then becomes trivial to answer.

(figure of federated databases here).

The federated database system extends the peer to peer systems discussed
earlier and provides a direct means of solving the problems of database
fragmentation by subdiscipline. Since the requirements for being a
`node' in the federation are minimal, individual, local servers work
seamlessly with institutional servers and large, national servers to
take advantage of all available storage and bandwidth of the
participating servers --- a promising way to solve the problems posed by
the ``big data'' of contemporary science (eg. one articulation by \cite{charlesCommunityDrivenBigOpen2020},).While mappings between
schemas are not magical and require work, they provide a single point of
mediation between the data formats of different disciplines. Federation
gets us the best of both worlds: the flexibility and domain-specific
tools of subdisciplinary databases with the availability of
domain-general databases. The radical autonomy of federated systems
dramatically lowers the barriers to standardization: rather than
requiring everyone to do \emph{the same thing in the same way} and
fundamentally change how they do things, researchers need to just build
the bridges to connect their existing systems to the federated standard.
These bridges can be created gradually. Since nodes in a federated
system are free to choose whether they connect to others, there do not
need to be mappings between \emph{all} types of data in a federation,
and there is no need for creating the oft-fabled \emph{``one true
standard''} for all data. Researchers that are interested in interfacing
their data with others are strongly incentivized to write the mappings
that permit it, and so they can emerge as they are demanded. Researchers
are also given far more control over their own data than is afforded by
traditional databases: it is entirely possible to have fine-grained
permissions controls that allow researchers to share only the data they
want to with the rest of the system while still taking advantage of, for
example, locally federated servers that make their data available to
other collaborating labs.

It's difficult to overstate how fundamentally a widely-adopted federated
database system would be for all domains of science: when designing a
behavioral experiment to study the circadian cycle, rather than relying
on rules of thumbs or a handful of papers, one could directly query data
about the sleep-wake cycles of animals recorded by field biologists in
their natural habitats, cross reference that with geophysical
measurements of daylight times and temperatures in those locations, and
normalize the intensity of light you plan to give your animals by
estimating tree-canopy coverage from LIDAR data from the geographers.
One could make extraordinarily biophysically realistic models of neural
networks by incorporating biophysical data about the properties of ion
channels and cell membranes, tractography data from human DTI fMRI
images, and then compare some dynamical measurement of your network
against other dynamic systems models like power grids,
telecommunications networks, swarming ants, and so on.
Seemingly-intractable problems like the ``file drawer'' problem simply
dissolve: null results are self-evident and don't \emph{need}
publication when researchers asking a question are able to see it
themselves by analyzing all previous data gathered. Without
exaggeration, they present the possibility of making \emph{all}
experiments multidisciplinary, making use of our collected human
knowledge without disciplinary barriers. Indeed nearly all scientific
literature \href{https://freeread.org/ipfs/}{is already available on a
federated database system} to anyone with an internet connection ---
arguably the largest expansion of scientific knowledge accessibility
ever.

The fundamental tradeoff between centralized and decentralized database
systems is that of flexibility vs.~coherence: centralized systems can
simply enforce a single standard for data and assume that everything it
works with will have it. Federated systems require some means of
maintaining the mappings between schemas that allow their fluid
translation. They also require some means of representing and
negotiating data that is unanticipated by existing schemas. The fine
details of implementing a federated database system are outside the
scope of this paper, but we will return to a means of distributed
maintenance of mappings between schemas by taking advantage of semantic
web technologies in \protect\hyperlink{shared-knowledge}{shared
knowledge}. Before we do though, we need to discuss the shared tools to
analyze and generate the data for the system in this section.

!! make sure to talk about datalad and DANDI!! https://www.datalad.org/

!! federated systems let us bridge the gap between localized server
technology like datajoint and mass server technology like databases. If
you let people federate at a local scale to share data between an
institute, a consortium, etc. and then let those things scale to
federate together you have a plausible means by which slowly a
generalized database system could be accumulated over time.

!! lots of times this has been proposed before \cite{
simaEnablingSemanticQueries2019,
djokic-petrovicPIBASFedSPARQLWebbased2017,
hasnainBioFedFederatedQuery2017}

\hypertarget{shared-tools}{%
\subsection{Shared Tools}\label{shared-tools}}
 
!! talk about Plugin Oriented Programming (pypi POP page) as a design
philosophy

If we're building infrastructure to allow us to build on each other's
labor by sharing data, why not do the same for the tools that analyze
and collect the data while we're at it? The benefits of distributed
infrastructure that allow us to preserve our collected labor and
knowledge compound when applied in multiple domains. The benefits of
shared data, analytical, and experimental infrastructure are far more
than the sum of their parts. Each is useful on its own, but as
additional components of the system are developed they make the
incentive to develop the rest even stronger \textless- this para is
dogshit. rewrite with a clear head.

This section will be relatively short as I feel like a shared analytical
framework is relatively uncontroversial, just a matter of putting labor
in the right place. I also don't want to give the impression of
self-promotion, as I have spent the last several years designing an
\href{https://docs.auto-pi-lot.com}{experimental framework}, autopilot.
I will discuss it because, unsurprisingly, I designed it based on the
same thoughts that have since developed into this paper, but I want to
be clear that as with the rest of the paper, my focus is on the
\emph{kind} of tools we need rather than promoting one specific tool.

\hypertarget{analytical-framework}{%
\subsubsection{Analytical Framework}\label{analytical-framework}}

The first natural companion of shared data infrastructure is a shared
analytical framework. A major driver for the need for everyone to write
their own analysis code largely from scratch is that it needs to account
for the idiosyncratic structure of everyone's data. Most scientists are
(blessedly) not trained programmers, so code for loading and negotiating
loading data is often intertwined with the code used to analyze it, so
it is often difficult to adapt another lab's analysis code for use in
other contexts. If instead neuroscientists had all their data in a
standardized format, then it would be possible to write an analysis
method once and allow the rest of the community to benefit from it.

A shared analytical framework should be

\begin{itemize}

\item
  \emph{modular} - Rather than implementing an entire analysis pipeline
  as a monolith, the system should be broken into minimal, composable
  modules. The threshold of what constitutes ``minimal'' is of course to
  some degree a matter of taste, but the overriding design principle
  should be to minimize the amount of duplicated labor. Rather than
  implementing a ``peri-stimulus time-histogram'' module, we should
  implement a ``binning'' module for counting spikes, connect it to an
  ``alignment'' module that splits the recording into chunks aligned at
  the stimulus onset, and so on. Higher-order analysis methods are
  relatively trivially composed from component parts, but extracting
  component parts from a frankenstein do-everything script is not. I
  expect this point to be relatively uncontroversial as it is a general
  principle of program design.
\item
  \emph{deployable} - For wide use, the framework needs to be easy to
  install and deploy locally and on computing clusters. The primary
  obstacle is dependency management, or making sure that the computer
  has everything needed to run the program. Anecdotally, more than the
  complexity of using the package itself, the primary barrier for
  nonprogrammer scientists using a particular software package is
  managing to get it installed. Luckily containerization and package
  management is a widespread and increasingly streamlined practice, so I
  expect this too to be uncontroversial.
\item
  \emph{pluggable} - The framework needs to provide a clear way of
  incorporating external analysis packages, handling their dependencies,
  and exposing their parameters to the user.
\item
  \emph{reproducible} - The framework should separate the
  \emph{parameterization} of a pipeline, the specific options set by the
  user, and its \emph{implementation}, the code that constitutes it.
  Implicit in a modularly constructed analysis framework is the notion
  of a ``pipeline,'' or a specification of a tree (or, specifically, a
  \href{https://en.wikipedia.org/wiki/Directed_acyclic_graph}{DAG}) of
  successive stages that process, merge, or split the data from the
  previous stage. The parameterization of a pipeline should be portable
  such that it, for example, can be published in the supplementary
  materials of a paper and reproduced exactly by anyone using the
  system.
\end{itemize}

Thankfully, \href{https://datajoint.io/}{DataJoint} already does most of
this, and is expanding its modularity with its recent
\href{https://github.com/datajoint/datajoint-elements}{Elements}
project.

!! need to revisit this in light of the paper: \cite{yatsenkoDataJointElementsData2021},Thoughit currently uses a
\href{https://docs.datajoint.io/python/admin/1-hosting.html}{MySQL},
relational database as its backend, extending it to incorporate with the
peer to peer database system described above would be an early, concrete
development goal for this program. I have heard rumors they are
considering adopting a decentralized traditional relational database
like \href{https://www.cockroachlabs.com/product/}{CockroachDB}, which
is not the same thing as a p2p federated semantic database system as I
describe here, but is certainly a step in that direction. The rest is in
the minutiae of normal software development, as well as building a user
interface and collaboration platform for curation and management of
shared pipelines. Thank you DataJoint team for making this section so
simple.

The combined benefits of a unified data sharing and analytical system
have a far greater reach than just saving redundant development time:

Papers published with a concise, inspectable description of their
analytical pipeline sidestep the vagueries of methods section prose and
allow widescale independent replication of published analyses. A system
of documenting and discussing the countless hyperparameters and
preprocessing tricks, often as much art as science, could operate as a
means of implementing the countless papers describing best practices in
analysis. If made easily expandable, so that the developers had a clear
way to integrate their tools, access to the state of the art in analysis
would be radically democratized, rather than limited to those with
finely-tuned twitter feeds and patience to wade through seas of errors
and stackexchange posts to get them to work.

A common admonishment in cryptographically-adjacent communities is to
``never roll your own crypto,'' because your homebrew crypto library
will never be more secure than reference implementations that have an
entire profession of people trying to expose and patch their weaknesses.
Bugs in analysis code that produce inaccurate results are inevitable and
rampant \cite{millerScientistNightmareSoftware2006,
soergelRampantSoftwareErrors2015, eklundClusterFailureWhy2016a,
bhandarineupaneCharacterizationLeptazolinesPolar2019}, but
impossible to diagnose when every paper writes its own pipeline. A
common analysis framework would be a single point of inspection for
bugs, and facilitate re-analysis and re-evaluation of affected results
after a patch.

Perhaps more idealistic is the possibility of a new kind of scientific
consensus. Scientific consensus is subtle and elusive, but to a very
crude approximation two of the most common means of its expression are
review papers and meta-analyses. Review papers make a prose argument for
a consensus interpretation of a body of literature. Meta analyses do the
same with secondary analyses, most often on the statistics reported in
papers rather than the raw data itself. Both are vulnerable to sampling
problems, where the author of a review may selectively cite papers to
make an argument, and meta-analyses might be unable to recover all the
relevant work from incomplete search and data availability. Instead if
one could index across all data relevant to a particular question, and
aggregate the different pipelines used to analyze it, it would be
possible to make statements of scientific consensus rooted in a full
provenance chain back to the raw data.

More fundamentally, a shared data and analysis framework would change
the nature of secondary analysis. Increasing rates of data publication
and the creation of large public datasets like those of the Allen
Observatory make it possible for metascientists and theoreticians to
re-analyze existing data with new methods and tools. There is now such a
need for secondary analysis that the NIH, among other organizations, is
providing
\href{https://grants.nih.gov/grants/guide/rfa-files/rfa-mh-20-120.html}{specific
funding opportunities} to encourage it. Secondary analyses are still
(unfortunately) treated as second-class research, and are limited to
analyzing one or a small number of datasets due to the labor involved
and the diversity of analytical strategies that makes a common point of
comparison different. If, say some theoretician were to develop some new
analytical technique that replaced some traditional step in a shared
processing pipeline, in our beautiful world of infrastructure it would
be possible to not only aggregate across existng analyses, as above, but
apply their new method across an entire category of research.

In effect, analytical infrastructure can at least partially ``decouple''
the data in a paper from its analyis, and thus the interpretations
offered by the primary researchers. For a given paper, if it was
possible to see its results as analyzed by all the different processing
pipelines that have been applied to it, then a set of observations
remains a living object rather than a fixed, historical object frozen in
carbonite at the time of publication. In addition to statements of
consensus that can programmatically aggregate \emph{existing} results as
described by the primary researchers, it also becomes possible to make
\emph{fluid} statements of consensus, such that a body of data when
analyzed with some new analysis pipeline can yield an entirely
\emph{new} set of outcomes unanticipated by the original authors. I
think many scientists would agree that this is how an ideal scientific
process would work, and this is one way of dramatically lowering the
structural barriers that make it deviate from that ideal.

I'll give one more tantalizing possibility here: at the point when we
have a peer-to-peer federated system of data-sharing servers integrated
with some easily deployable analysis pipelining framework, then we also
get a distributed computing grid akin to
\href{https://foldingathome.org/}{Folding@Home} where users donate some
of the computing power of their servers to analyze pieces of some large
analysis job with very little additional development.

\hypertarget{experimental-framework}{%
\subsubsection{Experimental Framework}\label{experimental-framework}}

On the other side of data from its analysis are the tools used for its
collection. A unifying experimental framework is seemingly a different
kind and scale of complexity compared to a unifying data framework.
\emph{Everyone needs completely different things!} I have previously
written about the design of a generalizable, distributed behavior
framework in section 2, and about one modular implementation in section
3 of \cite{saundersAutopilotAutomatingBehavioral2019},,and so I
will first abbreviate and extend the discussion found there and then
consider the role of an experimental framework in broader scientific
infrastructure. I designed
\href{https://docs.auto-pi-lot.com}{Autopilot} with many of the same
fundamental motivations as I articulate here, so being dredged from the
same well it should be far from surprising that I see it as a natural
example. My intention is not as a self-serving advertisement for
\emph{everyone to use my software,} but to use it as an \emph{example}
of the \emph{kind} of tool that I think would fit a particular role in a
broader set of scientific infrastructure (!! redundant, pick a framing).

I first want to clarify what i'm talking about as an `experimental
framework' -- not talking about projects \textbf{that we love} like open
ephys/etc that develop specific hardware. Those are strictly
complementary (and should be given more resources!) I'm talking about
something to unify them, to combine the excellent pieces that implement
differnt parts of experiments into a unified system.

The most basic requirement of a piece of shared experimental
infrastructure is that it must be capable of expressing and being
adapted to \textbf{perform any experiment.} The ``any'' there is a
hard-ish ``any,'' the reason for which should become clearer soon. At an
extremely abstract level, this means that the framework needs to be able
to \textbf{control potentially high numbers of independent hardware
components,} record measurements from them, and coordinate them together
in some logical system that constitutes a ``task'' (or more broadly an
``experiment''). In order to be widely adoptable, it needs to be able to
\textbf{integrate with the instrumentation that researchers already use}
rather than requiring researchers to reoutfit their entire rigs. That
means, in turn, that it needs to provide a clear means for users to
\textbf{extend its functionality} and contribute their extensions to the
framework. At the same time as providing a clear entrypoint for
researcher-developers to interact with the code, it needs to provide a
\textbf{simple user inferface} so that regular use doesn't require
extensive programming knowledge. In other words, if it ain't usable by
everyone, it ain't infrastructure, and the same can be said for expense:
it must be \textbf{inexpensive to implement.} Finally, it needs to be
purpose-built for \textbf{reproducibility and replication} by preserving
a full chain of \textbf{provenance} across the wandering path of
parameter tuning and experimental design in a clear,
\textbf{standardized data format} and providing a means of
\textbf{replicating experiments} even in rigs that are only an
approximate match to the original.

Autopilot attempts to achieve these lofty goals by embracing a
distributed, modular architecture. Autopilot is built as a system of
modules that each represent fundamental parts of experiments in general:
hardware control, stimulus generation, data management, and so on.
Everything is networked, so everything can talk to anything, even and
especially across computers: in practice this means that it is capable
of coordinating arbitrary numbers of experimental hardware components by
just \emph{using more computers.} It is built around the Raspberry Pi, a
low-cost single-board computer with an enormous support community and
library of off-the-shelf components, but can be used on any computer.
Autopilot imposes few limitations on the structure of tasks and
experiments, but also gives users a clear means of defining the
parameters they require, the data that will be produced, how to plot it,
and so on, such that any task has a portable, publishable representation
that is not dependent on the local hardware used to implement it. Its
modular hierarchy already provides structure that makes it easy for
researchers to modify existing components to suit their needs, and some
of its co-developers and I are currently implementing a generalized
plugin system that will allow users to replace any component of the
system in such a way that their work can be made available and
referenceable by any other user of the system. Information about the
state of the system, the plugins used, the history of tasks and
parameters that an experimental subject experiences, are all obsessively
documented, and the data it produces is clean at the time of
acquisition. Portable task descriptions, referenceable plugins, and
exact documentation of provenance make Autopilot capable of facilitating
replication while still supporting extreme heterogeneity in its use. In
sum, we designed Autopilot to be flexible, efficient, and reproducible
enough for use as general experimental infrastructure.

When compared, the preceding reads as a rephrasing of the design
principles articulated in (!! link to section). Autopilot is of course
far from a finished project, and many of its design goals remain
aspirational due to the small number of contributors\footnote{I am the
  first to admit Autopilot's shortcomings, which I document extensively
  in its
  \href{https://docs.auto-pi-lot.com/en/latest/todo.html}{development
  roadmap} and github issues.}. I would be remiss in failing to mention
\href{https://bonsai-rx.org/}{Bonsai}, which I love and have learned a
lot from. I view Bonsai as a somewhat complementary project and would
one day love to merge efforts. The primary differences between Bonsai
and Autopilot, besides the massive and obvious difference in number of
users and maturity of the library, are a) Autopilot is written in
Python, a high-level programming language, and ``glues'' together fast,
low-level library, where Bonsai is written in C\#, which is also quite
fast but is comparatively less accessible to a broad number of users.
Relatedly, Autopilot's documentation describes how the library works
down to the lowest levels while Bonsai's is more focused on the user
level. b) Autopilot emphasizes communication between objects and their
use in a distributed architecture, while Bonsai provides an excellent
means of chaining objects together on a single system. c) Autopilot
makes comparatively more nudges, and provides a few more features for
making reproducible tasks and standardizing data. Again my intention is
not a self-serving advocacy for my software, but to say that Bonsai is
another extremely capable and widely-used system, and we need systems
\emph{like} them capable of serving the role in broader infrastructure
that I will turn to now.

In addition to the benefits of reduced duplication of labor and greater
access to the state of the art that runs through this whole argument, a
standardized experimental framework multiplies the benefits of the data
and analytical systems described previously.

When we talk about standardizing data, we talk in the parlance of
``conversion,'' but conversion is only necessary because reserachers
collect data in local, idiosyncratic formats. The reason researchers
rely on idiosyncratic formats is that it is far from straightforward to
directly collect data from their heterogeneous tools in a standardized
format. The need for data conversion leaves an airgap between the ideal
of universal data access and its labor-intensive practical reality: only
those that are most ideologically committed and have enough resources to
convert \& share their data will do so. We could (and should) lessen the
chore of data conversion with continued development of intuitive
conversion tools, but an experimental framework that collected data that
was \emph{clean at the time of acquisision} then we could shortcircuit
the need for conversion altogether. It would also completely dissolve
the need for researchers to interact with the peer-to-peer sharing
system described previously by automatically dumping standardized data
directly into it. In short, an experimental framework could make all the
steps between collecting and sharing data completely seamless, and by
doing so make the dream of universal data availability possible.

Neuroscience has made substantial progress standardizing an ontology of
common terms for cells, chemicals, etc. (see the
\href{https://github.com/SciCrunch/NIF-Ontology}{Neuroscience
Information Framework's Ontology}) but an ontology for the many minute
parameters that define a behavioral experiment's operation has proven
elusive. Creating a standardized language for expressing and
communicating behavioral experiments is the object of one of the
Neurodata Without Borders
\href{https://archive.org/details/nwb-behavioral-task-wg}{working
groups}, in collaboration with the
\href{https://archive.org/details/beadl-xml-documentation-v-0.1}{BEADL}
project, and they've done admirable work there. They have an in-progress
terminology for certain parameters like \texttt{Reward}, \texttt{Guess},
etc., as part of a state-machine (!!define in margin) based
representation of a task. The model of standardization would then be to
define some extensible terminology, and then either build some software
that implements the state machine descriptions of tasks or else ask
existing software developers to incorporate them in their systems.

This path to standardization has many attractive qualities, like the
formal verification possible with state machines, but may have trouble
reaching universal adoption: at even modest complexities, experiments
that are simple to explain in prose can be awkward and complicated to
express as state machines (eg. section 3.1 in \cite{saundersAutopilotAutomatingBehavioral2019},,though the proposed
\href{https://statecharts.github.io/}{statecharts} model is a bit
friendlier than traditional state machines). If it is difficult to
express a particular feature of some experiment in some formalism, and
easier to implement it as some external software, unintegrated with the
behavioral framework, then much of the appeal of standardization is
lost.

\textbf{bigg redundancy from here\ldots{}}

Uniform standardization is desirable in the circumstances where it is
possible, but the scale of variability in the parameters and designs of
behavioral neuroscience experiments is truly on a different scale than
the already-perplexing case of measurement data standardization. For
example, a standard experiment in our lab implemented in Autopilot can
be fully described by the parameters that define the experimental
protocol itself, and those that parameterize the raspberry pi and the
experimental hardware
(\href{https://gist.github.com/sneakers-the-rat/eebe675326a157df49f66f62c4e33a6e}{here
they are}). The training protocol consists of 7 shaping stages that
gradually introduce a mouse to a fairly typical auditory categorization
task, each of which includes the parameters for at most 12 different
stimuli per stage, probabilities for presenting lasers, bias correction,
reinforcement, criteria for advancing to the next stage, etc. The rest
of the parameterization includes details for configuring, calibrating
and operating the rest of the system -- and this is the minimal set of
parameters for replicating this experiment that excludes all the
defaults, implicit behavior, and well, the rest of the system. For this
one relatively straightforward experiment, in one lab, in one
subdiscipline, there are 268 different parameters. It's not really about
the \emph{number} of parameters per se, but their unpredictability: one
needs to parameterize every electrode on a neuropixel probe, but they
are shared across a comparatively small number of things of their kind.

Asking people to change the entire way they think about, describe, down
to the very mental model that they use to think about it is actually a
huge ask. Even if some reasonable standardized lexicon was proposed, it
will face the same difficulties as, well, normal lexicons: there is no
neutral `name' for anything, and any word is dependent on the way we
conceptualize its use and meaning. This isn't woo-woo unknowability
shit: one person's sensory response latency is another person's time of
delayed gratification suppression. Even assuming that, getting everyone
to start re-expressing all their experiments in a probably very
different way than they have been thinking about them for 20-30 years
with all the entrenched hardware decisions made over that time is just
rounding the bend ready to beat u up for ur lunch money.

Another, complementary way of approaching this problem is to focus on
giving people a way to express themselves in a `safe' environment, focus
on the way they \emph{use} them rather than try to define all of them
a-priori. sorta like lameguage lmao.

a behavioral framework designed for reproducibility, that preserves a
complete history of task parameters as well as the code that uses them,
solves both the problems of external inspection and replication without
needing to prescribe a specific formalization or uniform ontology. It
doesn't matter \emph{what} terms you use if it's trivial to see
\emph{how} they're used. Importantly, this strategy punts on the goal of
interoperability, but does not forsake it: we will revisit standardized
ontologies in the next section. Asking large numbers of people to change
the way that they think about their experiments and the words they use
to describe them is, ultimately, a pretty big ask. Providing people a
tool that allows themselves to express themselves in whatever form is
natural to them and make their terminology meaningful by preserving its
context might be easier. (put people in the same system and give them a
space to express the terms they use and let them standardize among
themselves rather than imposing.)

\textbf{\ldots{} to here}

Replication is seriously hard. designing a software system that's smart
enough about the division between the logical structure of the task and
the implementation is seriously hard. the raspi is general purpose
enough that was can incorporate pretty general purpose hardware control
systems with nontrad components as well, so it balances being an
approachable ``start from somewhere'' (actually in a really good place)
with general still byo-hardware. replication needs to basically be
incorporated from the ground up, as most behavioral packages that exist
tend to rely on local script files that are still labor-intensive to
create and are rarely shared, because they're not really intended to be
made sharable. \textless\textless{} point im' trying to make here is
that it can't be an afterthought, the ways that it's easy to go wrong.

But for systems that do link code to a portable task description, where
the documentation for each parameter is also good (like wat if that
documentation was linked to the semantic wiki\ldots{} return to in next
section), then it is entirely possible to download a system that you
point to whatever parts you have around and let er rip. (this doesn't
address the technical complexity, but that's also a tease for the next
section).

It is already occasionally possible to follow the trail of provenance
back to some experimental code, but when all code is developed
independently, is any of it reliable \cite{wallReliabilityStartsExperimental2018},?Like bugs in analytical
software, bugs in experimental control software are likely rife, but
unless they are present in the few pieces of commonly used open-source
software they are almost entirely undiagnosable. Conversely, maybe more
positively, a shared experimental framework gives a place to gather
reference implementations of the many common algorithms, routines, and
hardware controllers used in neuroscientific experiments. (!! the tiny
details matter, but they almost never make it into methods sections. eg.
we use bias correction methods, but the way we do it might be different
than the way you do it. people do lots of great work optimizing over
different training regimens, but that usually gets left as text.
Algorithms for hardware control and sensor fusion are split across a
zillion adafruit libraries, and they aren't modularized or split up or
even documented that well)

As an example, intertial motion sensors (IMUs) are an increasingly
common tool for neuroscientists interested in studying unrestrained,
freely moving behavior. In our case, we were working with
\href{https://web.archive.org/web/20210127212527/https://www.sparkfun.com/products/13944}{an
IMU} with three, 3-dimensional sensors: an accelerometer, gyroscope and
magnetometer. The raw signals from these IMUs (linear acceleration,
angular velocity) are rarely useful on its own, and researchers are
usually after some derived value like orientation, position, etc. Since
the readings are also noisy, these transformed signals typically rely on
some \href{https://en.wikipedia.org/wiki/Sensor_fusion}{sensor fusion}
algorithms to condition and combine them. We were interested in
measuring ``absolute'' geocentric vertical velocity to control a
motorized platform in a closed-loop experiment (as I described in my
\href{https://neuromatch.io/abstract/?submission_id=recI5D0QaJ857Y4JI}{NMC3}
\href{https://youtu.be/l2K0l4ec0Xw}{talk}). Adafruit provides a
\href{https://github.com/adafruit/Adafruit_CircuitPython_LSM9DS1/blob/master/adafruit_lsm9ds1.py}{basic
Python} library to control the IMU, but it was relatively undocumented,
slow, and didn't expose all the functionality of the chip, so we
\href{https://web.archive.org/web/20210421223148/https://docs.auto-pi-lot.com/en/parallax/autopilot.hardware.i2c.html\#autopilot.hardware.i2c.I2C_9DOF}{adapted
it to Autopilot}. We were able to find a number of whitepapers that
described a sensor fusion algorithms, but no implementations. The
algorithm we eventually landed on uses a
\href{https://en.wikipedia.org/wiki/Kalman_filter}{Kalman filter} to
combine accelerometer and gyroscope readings to estimate orientation
\cite{abyarjooImplementingSensorFusion2015a},.In this case were
lucky to find Roger Labbe's excellent
\href{https://github.com/rlabbe/filterpy}{filterpy} library \cite{
labbeKalmanBayesianFilters2020, labbeRlabbeFilterpy2021}, and with a
few performance and syntax tweaks were also able to
\href{https://web.archive.org/web/20210421223300/https://docs.auto-pi-lot.com/en/parallax/autopilot.transform.timeseries.html\#autopilot.transform.timeseries.Kalman}{adapt
it to autopilot}, extended it to implement the
\href{https://web.archive.org/web/20210421212747/https://docs.auto-pi-lot.com/en/parallax/autopilot.transform.geometry.html}{orientation
transformation}, and built it into the
\href{https://github.com/wehr-lab/autopilot/blob/6843c0e7b6e2bfb4c35e2f7c41972336765feabd/autopilot/hardware/i2c.py\#L469-L501}{IMU
Object}. (!! - go back through and give names to each of the objects and
methods for reference below)

OK cool so you programmed an accelerometer, what's the big deal? First,
from the developer's perspective: we needed to implement some
\texttt{hardware} object and teach it about some geometric
\texttt{transform}ation. The autopilot \texttt{hardware} and
\texttt{transform} modules give a clear place to implement both. The
minimal expected structure of these modules make it straightforward to
adapt existing code to the library, but we can also copy and modify (or,
``inherit from'') some existing hardware object to avoid having to write
basic operations from scratch (eg. the I/O operations) and extend their
functionality (eg. they are now networked, can take advantage of
autopilot's unified logging system, etc.). To the degree that the
framework is widely adopted, it gives credit to and provides a direct
means of making the algorithms and tools they develop available to
users. (!! they don't even need to integrate with the system wholesale,
just expose some API and write a quick wrapper for this, we're wrking on
it!!) This is, in some sense, the essence of what i mean by a behavior
``framework'' --- a minimal ``spanning set'' of rules for how the system
works that gives clear points of extension.

From the user's perspective: We could have implemented the sensor fusion
algorithm and geometric transform directly ``in'' the \texttt{IMU}
hardware object, but instead we separated them out as several
independent \texttt{transform} objects. Rather than extending the
functionality of a single hardware object, we instead gained several
basic algorithms. Their generality is \emph{noncoercive} --- The problem
of getting absolute orientation from an IMU is solved for
\emph{everyone}, even those that don't want to adopt any other part of
the system. The rotation algorithm is generic and modular: it can be
used as just a trigonmetric transformation of accelerometer readings
without a Kalman filter, incorporate gyroscopic readings, or use an
entirely different timeseries filter altogether. By integrating them in
an existing library of \texttt{transform} objects, they are made
combinatorically more useful --- so far all I have discussed has been a
means of estimating orientation, we still need to \emph{use} that
orientation estimate to extract ``absolute'' vertical acceleration.

In autopilot, we can express the rest of what we need as a series of
\texttt{transform} objects that can be added together with \texttt{+}
and \texttt{+=}:

!! add formatted code example

!! add example of adding DLC to position estimate? yes. to show how
things don't need to be built into autopilot, just given some API, as
was done with deeplabcut, at the end of the developer section

From the often-overlooked perspective of some downstream ``reader'':
when everything is integrated into an extensible experimental framework,
complete retrospective provenance becomes possible. Autopilot
exhaustively logs all local parameters like hardware configuration, as
well as references to all versions of all code used to generate a
dataset \emph{in the dataset itself} automatically. A reader can then
trace all the data presented in a paper back through a standardized
analysis pipeline to the raw data, and then continue to inspect every
part of how it was generated. Since the code is all available by
default, it becomes possible to audit experimental code on a broad
scale: if a reader were to find a bug, they could raise an issue, patch
it --- and flag all datasets that were effected.

At this point we have largely closed the loop of science: starting with
standardized data, shared in a scalable p2p system, with some federated
interface structure, through modularized analysis parts, published
alongside the means to directly reproduce the experiment and re-generate
the data\ldots{} and when we start considering these technologies as an
ensemble some things that truly sound like science fiction compared to
scientific reality start to become possible. In addition to allowing all
of the above features of standardized output data being cross-indexable,
what about making the literal fine-grained parameters a way of indexing
The All Knowledge Base (go back to previous section and make clearer
that only the output data is indexable)? Doing a simultaneous
optimization over all of our parameters is basically impossible, and we
have all these heuristics for hopping and skipping over it, but what if
the behavioral system could query all other times the experiment has
been performed, cross reference with published outcome data from the
parameterization, and recommend the optimal parameterization for
whatever you are studying? The compounding nature of making systems that
preserve and respect the diversity of labor to make it coproductive is
tectonic: at every stage, from implementation to tweaking, to
understanding a science with appropriate infrastructure would move at
light speed compared to the way we do it now.

\ldots{} the major part that's missing is some means of negotiating our
schemas and data \ldots{} transition to next section

\hypertarget{shared-knowledge}{%
\subsection{Shared Knowledge}\label{shared-knowledge}}

!! https://www.dbpedia.org/

!! why is public trust in scientists so low? could it be that there is
an alternative to scientists seeing themselves as cloistered experts?
re: cold war peer review paper

The (part of the system that's most needed and potentially
transformative) is a system of scientific communication.

Except for certain domain-specific exceptions, the scientific
communication system consists of the two ancient monoliths groaning with
the dust of their obsolescence: the dead and static papers of the
traditional journal system, and the ephemeral halo of insider knowledge
shared at conferences. The remainder of the gigantic overflowing franzia
bag of scientific discourse is funelled ingloriously onto
Twitter\footnote{no citation needed, right? if there is some other
  bastion of scientific discourse i would love to know about it.} ---
and it \emph{sucks}.

Since the advent of the contemporary journal system, communication
technology has been stripped to its very atoms and rebuilt --- and it
has managed to dig in and \emph{persist} while all the letterman jackets
and beatniks of its era have become vape teens on tiktok. A
reconsideration of the entire scientific publishing system is strictly
out of scope for this paper, but the communication system I will
describe exists in the gaps of need it leaves unfilled. Criticisms of
the scientific communication system typically start by imaginging much
of the contemporary journal system as etched as fact on the face of
reality, and tweaking at a few of its more ticklish knobs (eg. \cite{heesenPeerReviewGood2020},).Instead let's try it the other way: to
trace the outlines of how a scientific communication system
\emph{should} work, given the basis of holistic infrastructure described
so far. I will argue that a communication system, and more specifically
the community it supports, is the blood that must pump through any of
these digital systems that aspire to call themselves infrastructure. To
arrive at a proposed form for a system, I'll start by laying the basic
axes of communication technology, and then load the scales with the
empirical girth of the largest knowledge systems that have ever existed:
Wikipedia and internet piracy.

There simply isn't a place to have longform, thoughtful, durable
discussions about science. The direct connection between the lack of a
communcaition venue to the lack of a way of storing technical,
contextual knowledge is often overlooked. Because we don't have a place
to talk about what we do, we don't have a place to write down how to do
it. Science needs a communcation platform, but the needs and constraints
of a scientific communication platform are different than those
satisfied by the major paradigms of chatrooms, forums etc. By
considering this platform as another infrastructure project alongside
and integrated with those described in the previous sections, its form
becomes much clearer, and it could serve as the centerpiece of
scientific infrastructure.

I will argue that a semantic wiki should be the major piece of durable
information storage, and that it should be supported by a forum system
for discussion.

!! description of its role as a schema resolution system -- currently we
implement all these protocols and standards in these siloed, centralized
groups that are inherently slow to respond to changes and needs in the
field. instead we want to give people the tools so that their the
knowledge can be directly preserved and acted on.

!! descrption of its role as a tool of scientific discussion --
integrated with the data server and standardized analysis pipelines, it
could be possible to have a discussion board where we were able to pose
novel scientific questions, answerable with transparent, interrogatable
analysis systems. Semantic linking makes the major questions in the
field possible to answer, as discussions are linked to one another in a
structured way and it is possible to literally trace the flow of
thought.

!! let's tour through wikipedia for a second and see how it's organized.
Look at these community incentive structures and the huge macro-to-micro
level organization of the wiki projects. The infinitely mutable nature
of a wiki is what makes it powerful, but the SaaS wikis we're familiar
with don't capture the same kind of `build the ground you walk on'
energy of the real wiki movement.

!! what's critically different here between other projects is that we
are explicitly considering the incentives to join each of these efforts,
and by integrating them explicitly, each of them is more appealing. so
while there are lots of databases, lots of analysis systems, lots of
wikis, and so on, there aren't many that are linked with one another
such that participating in one part of the system makes the rest of the
system more powerful as well as makes it more useful to the user.

\hypertarget{axes-of-communication-systems}{%
\subsubsection{Axes of Communication
Systems}\label{axes-of-communication-systems}}

\hypertarget{semantic-wikis---technical-knowledge-preservation}{%
\subsubsection{Semantic Wikis - Technical Knowledge
Preservation}\label{semantic-wikis---technical-knowledge-preservation}}

\cite{kamelboulosSemanticWikisComprehensible2009},!!Read and cite! \cite{classeDistributedInfrastructureSupport2017},!!the word for communally curated schemas is
https://en.wikipedia.org/wiki/Folksonomy

!! \cite{goodSocialTaggingLife2009},!!wikibase can do federated SPARQL queries https://wikiba.se/ - and has
been used to make folksonomies https://biss.pensoft.net/article/37212/

\begin{leftbar}
I can see my bank statements on the web, and my photographs, and I can
see my appointments in a calendar. But can I see my photos in a calendar
to see what I was doing when I took them? Can I see bank statement lines
in a calendar? https://www.w3.org/2001/sw/
\end{leftbar}

!! lots of scientific wikis -
https://en.wikipedia.org/wiki/Wikipedia:WikiProject\_Molecular\_Biology/Genetics/Gene\_Wiki/Other\_Wikis
-
https://en.wikipedia.org/wiki/Wikipedia:WikiProject\_Molecular\_Biology/Genetics/Gene\_Wiki

\hypertarget{semantic-wikis---schema-resolution-communication-platform}{%
\subsubsection{Semantic Wikis - Schema Resolution \& Communication
platform}\label{semantic-wikis---schema-resolution-communication-platform}}

!! bids is doing something like this https://nidm-terms.github.io/

!! interlex

\begin{leftbar}
The Semantic Web is about two things. It is about common formats for
integration and combination of data drawn from diverse sources, where on
the original Web mainly concentrated on the interchange of documents. It
is also about language for recording how the data relates to real world
objects. That allows a person, or a machine, to start off in one
database, and then move through an unending set of databases which are
connected not by wires but by being about the same thing.
https://www.w3.org/2001/sw/
\end{leftbar}

!! Semantic combination of databases in science are also not new \cite{cheungSemanticWebApproach2007,simaEnablingSemanticQueries2019}.
We need both though! semantic federated databases!

Part of what is missing and a place where we could learn from librarians
is the notion of governance over a knowledge schema. People have a lot
of trouble with NWB because they doubt if it could account for all the
idiosyncracies in the types of data that we have to represent. But
instead if we have a way of capturing all that thought and insight and
practical experience in a governance and decisionmaking structure then
we could flexibily work our way to a set of schemas that work for
everyone. Part of what needs to be done is to move from SQL queries to a
more expressive abstract system of schema creation that more people can
participate in -- that's what infrastructure building is, making things
that seem impossible or difficult routine. Practically, this can mean an
explicit versioning system that not only specifies different versions of
a data representation, but for every transition between state there is
some notion of making that transition in the data structure. (give
example of the subject upgrade system). If that was possible, then the
notion of data structure would entirely evaporate, best of both worlds.
we get everything and the game is over forever. This is also the
distinction between centralized and decentralized systems. we can just
make the changes and since they're done against a background of unified
intent and expression they can exist simulataneously, commune with one
another, while being forwardly productive as their contradictions are
resolved.

\hypertarget{linked-communication-platform}{%
\subsubsection{Linked communication
platform}\label{linked-communication-platform}}

We all hate science twitter, why does it exist?

\begin{leftbar}
Though frequently viewed as a product to finish, it is dynamic
ontologies with associated process-building activities designed,
developed, and deployed locally that will allow ontologies to grow and
to change. And finally, the technical activity of ontology building is
always coupled with the background work of identifying and informing a
broader community of future ontology users. \cite{bowkerInformationInfrastructureStudies2010},\end{leftbar}
good science community infra - https://www.zooniverse.org

\begin{leftbar}
Two essential features coordinate this information to better serve our
organizational decision-making, learning, and memory. The first is our
constellation of Working Groups that maintain and distribute local,
specialized knowledge to other groups across the network. {[}\ldots{]} A
second, more emergent property is the subgroup of IBL researchers who
have become experts, liaisons, and interpreters of knowledge across the
network. These members each manage a domain of explicit records (e.g.,
written protocols) and tacit information (e.g., colloquialisms, decision
histories) that are quickly and informally disseminated to address
real-time needs and problems. A remarkable nimbleness is afforded by
this system of rapid responders deployed across our web of Working
Groups. However, this kind of internalized knowledge can be vulnerable
to drop-out when people leave the collaboration, and can be complex to
archive. An ongoing challenge for our collaboration is how to archive
both our explicit and tacit processes held in both people and places.
This is not only to document our own history but as part of a roadmap
for future science teams, whose dynamics are still not fully understood.
\cite{woolKnowledgeNetworksHow2020},\end{leftbar}
importantly, semantic wiki can be accessed progrtammatically, so you
don't need to use the service and can build your own interface to it.

\begin{leftbar}
Relational database systems, manage RDF data, but in a specialized way.
In a table, there are many records with the same set of properties. An
individual cell (which corresponds to an RDF property) is not often
thought of on its own. SQL queries can join tables and extract data from
tables, and the result is generally a table. So, the practical use for
which RDB software is used typically optimized for soing operations with
a small number of tables some of which may have a large number of
elements.

RDB systems have datatypes at the atomic (unstructured) level, as RDF
and XML will/do. Combination rules tend in RDBs to be loosely enforced,
in that a query can join tables by any comlumns which match by datatype
-- without any check on the semantics. You could for example create a
list of houses that have the same number as rooms as an employee's shoe
size, for every employee, even though the sense of that would be
questionable.

The Semantic Web is not designed just as a new data model - it is
specifically appropriate to the linking of data of many different
models. One of the great things it will allow is to add information
relating different databases on the Web, to allow sophisticated
operations to be performed across them.
https://www.w3.org/DesignIssues/RDFnot.html
\end{leftbar}

in addition to a wiki, we need some conversational engine -- talk pages
are ok, but they're too fragmented and all hard to keep up to date with.
Realtime, chatlike interfaces don't preserve information well, so we
should use some intermediate medium like a forum or stack exchange that
allows conversations to be tagged and searched and sorted and organized.

Social incentive structure is huge here.

Compared to RDBMS https://www.w3.org/DesignIssues/RDB-RDF.html -- rather
than individual schemas, groupings of properties, we have
`relationships.' this example is good:

\begin{leftbar}
For example, one person may define a vehicle as having a number of
wheels and a weight and a length, but not foresee a color. This will not
stop another person making the assertion that a given car is red, using
the color vocabular from elsewhere.
\end{leftbar}

We're talking about a collaboration medium here\ldots{} we need a way of
organizing open questions in the field and discussing them in a
straightfoward way. Why is it that every scientist needs to figure out
their own completely gray-area way of discovering papers?

Bad APIs have killed projects with shitloads of funding like NWB and
IPFS https://macwright.com/2019/06/08/ipfs-again.html - usability needs
to be \emph{the first priority} - you can develop all the fancy shit
that you want, if no one can install and unse it in 10 minutes then it's
totally useless. This is why the community also has to be collaborative,
not just the technology, hends the shared governance idea\ldots{} ppl
note that IPFS has no economic model -- that's like true, because there
has to be some other incentive system for using it -- it makes your work
more powerful, it plugs you into a community, etc.
https://blog.bluzelle.com/ipfs-is-not-what-you-think-it-is-e0aa8dc69b

\hypertarget{credit-assignment}{%
\subsubsection{Credit Assignment}\label{credit-assignment}}

depth of linking is combinatoric -- if you have a paper ecosystem where
the numbers are linked to the data, and then the data is annotated, then
it's possible to index information across papers not just by textual
similarity metrics but on similarity of the structure of experiment and
data.

the work of maintaining the system can't be invisible, read \& cite \cite{ classeDistributedInfrastructureSupport2017,
bowkerInformationInfrastructureStudies2010}

\hypertarget{conclusion}{%
\section{Conclusion}\label{conclusion}}

\hypertarget{shared-governance}{%
\subsection{Shared Governance}\label{shared-governance}}

!! just make this a final note in the conclusion

In addition to like a wiki\ldots{} need some way of having conversations
and arguments about what means what. like some proposal system for
linking certain tags together or pointing one to the other\ldots so
shared knowledge and shared governance can be a fluid entity.

to avoid the coersion described in \cite{bietzCollaborationMetagenomicsSequence2009},wemust make any
metadata schema collaborative and mutually beneficial -- there is no
such thing as `required' data as long as we design a system that
preserves as much information as possible on collection, designing
infrastructure is an act of community trust.

Dont want to be prescriptive here, but that we can learn from previous
efforts like - https://en.wikipedia.org/wiki/Evergreen\_(software) , -
IBL, - etc.

\hypertarget{contrasting-visions-for-science}{%
\subsection{Contrasting visions for
science}\label{contrasting-visions-for-science}}

\hypertarget{the-worst-platform-capitalist-world}{%
\subsubsection{The worst platform capitalist
world}\label{the-worst-platform-capitalist-world}}

\hypertarget{what-we-could-hope-for}{%
\subsubsection{What we could hope for}\label{what-we-could-hope-for}}

As a break from doomsaying, imagine the positive vision of doing
neuroscience with all the power of basic infrastructure.

You have some new research question, and so you turn to the standard
Python (or whatever) library that allows you to query data from yours
and all other labs who share their data with this system. You're
immediately able to filter through to find all the recordings from a
particular subtype of cell in a particular region being exposed to some
particular set of stimuli across some particular manipulation. Since you
have access to decades of labor by thousands of scientists, even with
that complex filter you still find, say for the sake of having a round
cool-sounding number, a million recordings. Because they're all in some
standardized format, over the years a common analysis pipeline has been
developed, so you're also immediately able to perform the analyses to
confirm the hunch for your new question --- and it's time to implement
it.

You don't need to implement the whole thing from scratch because you can
check out a similar experiment from the standardized experimental
software framework, read the communally maintained documentation, make
the minor tweaks you need for your experiment, and you're off and
running. You need to build some brand new component, but you also have a
practical knowledge repository where other scientists working on similar
problems have described the basic components, circuits, and have even
uploaded some 3d-printable components for you to use. Because the
repository was designed for ease of use and has a robust system of
community incentives for contribution, as you build you document what
you learn, and when you're finished upload the schematics and write
instructions for your new component. The experimental software framework
was designed to incorporate custom components, so you extend some
similar hardware control code and integrate it with your experiment
without needing to resort to some patchwork system of TTL
synchronization pulses and serial port arcana.

You did it! Experiment Over! The experimental framework produces data
that is clean, annotated, and standardized at the time of acquisition,
and automatically integrates it with the analysis pipeline you built
when your experiment was just a budding baby hypothesis, so your
analysis is finished shortly after the experiment is. You have the
``auto-upload'' setting on, so without any additional effort your work
has been firehosing information back the global knowledge pool. You do a
pull request for the improvements you've made to the experimental
software, write the paper, and the loop is complete: a closed knowledge
system where nothing is wasted and everyone is more capable and
empowered by drawing from and contributing to it.

OK Here's the moment at the end of 2001.

end with the more radical vision --- science post papers. Information is
semantically organized, so it is possible to ask and answer questions
through the medium in which information is represented. Discussion
forums exist to describe particular kinds of questions, and a robust
discussion of primary scientific data is made possible. Scientists lost
their role as arbiters of all reality, but instead are just the comrades
closest to the questions, capable of answering open questions in the
community, able to design the experiments proposed.

The notion of the filedrawer problem dissappearing, we don't need to
publish null results when the data is all always available.

The fractal nature of provenance --- where if one can trace an
intellectual lineage through its data, one solves credit assignment as
centrality within a network.

High school biology classrooms are able to directly interface with the
fundament of science, open questions are directly open to students,

\hypertarget{references}{%
\section{References}\label{references}}


\end{document}