%!TEX root = ../prelims_main.tex
\section{Notes}

\subsection{Bailey \& Summerfield - 1980}
\label{note:cues}

A perceptual system in which the information for phonetic perception was a set of cues would have to incorporate three kinds of knowledge if it were to function successfully. It would have to know, first, which aspects of the acoustic signal are cues and which are not; second, it would need to possess a sensitivity to the pattern of cooccurrence of cues for each phone in its perceptual repertoire; third, it would need to appreciate the proper temporal coordination of the cues within each pattern. There is no reason, in principle, why a device could not be built to perceive phonetic identity from a substrate of acoustic cues, provided it was endowed with an articulatory representation sufficient to embody these three kinds of knowledge. However, we doubt that such a system could evolve in the natural world. For a species to acquire a knowledge of articulatory constraints, it would be necessary first that information specifying those constraints be available for the species, and second that the species possess a prior sensitivity to that information. The knowledge that a particular set of cues combine to indicate the presence of a given phone could be acquired in either of two ways. The identity of the phone could be specified independently of the set of acoustic cues, but this would hardly solve the problem and would preempt the need to evolve a sensitivity to the cues. Alternatively, the signal could specify directly both the identity of the cues and their temporal coordination, but then information in the signal that specified the coherence of its elements would, isomorphically, specify the articulatory event from which that coherence derived. However, the presence of this information about articulation in the signal, and a predisposition to register it on the part of the perceiver, would obviate the need for any internalized articulatory referent to mediate the acoustic-phonetic translation.

These considerations lead us to question the validity of equating the operational and functional definitions of an acoustic cue. A cue was defined operationally as a physical parameter of a speech signal whose manipulation systematically changes the phonetic interpretation of the signal. Although it is clear that perceptual sensitivity must exist to the consequences of manipulating a cue, it is not necessary to suppose that the cue is registered in perception as a discrete functional element.\cite{Bailey1980}
